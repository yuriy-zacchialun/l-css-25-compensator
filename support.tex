%%%%%%%%%%%%%%%%%%%%%%%%%%%%%%%%%%%%%%%%%%%%%%%%%%%%%%%%%%%%%%%%%%%%%%%%%%%%%%%%
%2345678901234567890123456789012345678901234567890123456789012345678901234567890
%        1         2         3         4         5         6         7         8

\documentclass[journal,twoside,web]{ieeecolor}
\usepackage{orcidlink}
\usepackage{lcsys}
\usepackage{amsmath,amssymb,amsfonts} % assumes amsmath package installed

\newtheorem{assumption}{Assumption}
\newtheorem{theorem}{Theorem}
\newtheorem{corollary}{Corollary}[theorem]
\newtheorem{remark}{Remark}
\newtheorem{lemma}{Lemma}
\newtheorem{proposition}{Proposition}
\newtheorem{definition}{Definition}
\newtheorem{problem}{Problem}

%%%%%%%%%%%%%%%%%%%%%%%%%%%%%%%%%%%%%%%%%%%%%%%%%%%%%%%%%%%%%%%%%%%%%%%%%%%%%%%%
\usepackage{mathtools}
%%%%%%%%%%%%%%%%%%%%%%%%%%%%%%%%%%%%%%%%%%%%%%%%%%%%%%%%%%%%%%%%%%%%%%%%%%%%%%%%
%%%%% For the references above equality and others relationships in equations
%%%%%%%%%%%%%%%%%%%%%%%%%%%%%%%%%%%%%%%%%%%%%%%%%%%%%%%%%%%%%%%%%%%%%%%%%%%%%%%%
\newcounter{relctr} %% <- counter for relations
\everydisplay\expandafter{\the\everydisplay\setcounter{relctr}{0}} %% <- reset every eq
\renewcommand*\therelctr{\alph{relctr}} %% <- label format
%%%%%%%%%%%%%%%%%%%%%%%%%%%%%%%%%%%%%%%%%%%%%%%%%%%%%%%%%%%%%%%%%%%%%%%%%%%%%%%%
\newcommand\labelrel[2]{%
  \begingroup
    \refstepcounter{relctr}%
    \stackrel{\textnormal{(\alph{relctr})}}{\mathstrut{#1}}%
    \originallabel{#2}%
  \endgroup
}
\AtBeginDocument{\let\originallabel\label} %% <- store original definition
%%%%%%%%%%%%%%%%%%%%%%%%%%%%%%%%%%%%%%%%%%%%%%%%%%%%%%%%%%%%%%%%%%%%%%%%%%%%%%%%

\title{Wireless Control With Channel State Detection And Message Dropout Compensation}
% \title{Linear Quadratic Regulation With Wireless Control Message Dropout Compensation}
% \title{Robust Linear Quadratic Regulation Over Polytopic Time-Inhomogeneous Markovian Channels Under Generalized Packet Dropout Compensation}

% \title{Robust LQR Over Polytopic Time-Inhomogeneous Markov Channels Under Generalized Packet Dropout Compensation}

\author{Yuriy \textbf{Zacchia Lun}\,\textsuperscript{\orcidlink{0000-0002-9408-8773}},\,\IEEEmembership{Member, IEEE}, Fortunato \textbf{Santucci}\,\textsuperscript{\orcidlink{0000-0002-0229-6277}},\,\IEEEmembership{Senior Member, IEEE},\\ and Alessandro \textbf{D'Innocenzo}\,\textsuperscript{\orcidlink{0000-0002-5239-0894}},\,\IEEEmembership{Member, IEEE}
\thanks{The authors are with the 
Department of Information Engineering, Computer Science and Mathematics (DISIM), University of L'Aquila, L'Aquila 67100, Italy. (e-mail addresses:
        {yuriy.zacchialun@univaq.it},\\ {fortunato.santucci@univaq.it}, and {alessandro.dinnocenzo@univaq.it})}%
}

\def\BibTeX{{\rm B\kern-.05em{\sc i\kern-.025em b}\kern-.08em
    T\kern-.1667em\lower.7ex\hbox{E}\kern-.125emX}}
\markboth{\journalname, VOL. XX, NO. XX, XXXX}
{Zacchia Lun \MakeLowercase{\textit{et al.}}: Wireless Control With Channel State Detection And Message Dropout Compensation}

\begin{document}

\maketitle
\thispagestyle{empty}
\pagestyle{empty}

\textit{Notation:} $\mathbb{R}$, $\mathbb{Z}^{+}$, and $\mathbb{Z}^{0+}$ indicate the sets of reals and positive and nonnegative integers. $\mathbb{R}^{n}$ indicates the $n$-dimensional Euclidean space. The symmetric positive definite and positive semidefinite matrices $Z$ are denoted by $Z\succ 0$ and $Z\succeq 0$. $Z^{\top}$ indicates the transpose of a matrix $Z$. Finally, $\mathbb{P}$ represents the probability of an event, and $\mathbb{E}$ denotes the expected value of a random variable.
\section{System Model and Problem Definition}\label{sec:model}
Consider a discrete-time linear system
\begin{equation}\label{eq:state}
        x_{k+1} = A x_{k} + B u_{k}^{} + w_{k},
\end{equation}
where $\forall k \!\in\! \mathbb{Z}^{0+}$, $x_k\!\in\!\mathbb{R}^{n_x}$ and $u_k^{}\!\in\!\mathbb{R}^{n_u}$ are the system state and control input to actuators, $A$ and $B$ are state and input matrices of appropriate size, and $w_k\!\in\!\mathbb{R}^{n_x}$ is a white Gaussian process noise having zero mean and covariance matrix $\Sigma_w$.

$\{\delta_k\}$ is a binary stochastic process that models the packet loss between the controller and the actuators.

\subsection{Polytopic Markovian channel model}\label{subsec:ptifsmc}
We model the dynamics of the control-packet loss process $\{\delta_k\}$ in the wireless propagation environment as an FSMC. 

This channel has a fixed finite set $\mathcal{S}$ of states $\{s_i\}_{i=1}^{N}$, each characterized by predefined signal-to-interference-plus-noise ratio (SINR) thresholds, which do not vary in time, so the state number remains constant, and each channel state has a specific packet dropout probability. 
A discrete-time Markov chain $\{\theta_k\}$ determines the active channel state, whose packet dropout probability and transition probabilities (TPs) depend on the SINR dynamics. So, we formally define the FSMC as follows.
For all $k\in \mathbb{Z}^{0+}$,
\begin{equation}\label{eq:p-ij}
    \mathbb{P}(\theta_{k} = s_j \mid \theta_{k-1} = s_i) = p_{ij} \geq 0,~ \sum_{j=1}^N p_{ij} = 1,
\end{equation}
\begin{align}\label{eq:p-delta}
\begin{aligned}
    \mathbb{P}(\delta_k = 1 \mid \theta_{k} = s_j) &= \hat{\delta}_{j}(k),\\
    \mathbb{P}(\delta_k = 0 \mid \theta_{k} = s_j) &= 1 - \hat{\delta}_{j}(k).    
\end{aligned}
\end{align}
For notational convenience, group %the
channel state TPs in the transition probability matrix (TPM) $P_{c}^{} \triangleq \left[p_{ij}\right]_{i,j=1}^{N}$
% \begin{equation}\label{eq:Pc}
%     P_{c}^{}(k) \triangleq \left[p_{ij}(k)\right]_{i,j=1}^{N}
% \end{equation}
and the  success (or, conversely, failure) probabilities of delivering a packet at time $k$ in the following matrices:
\begin{equation}\label{eq:Ps}
    P_{s}^{} \triangleq \left[p_{ij}\hat{\delta}_{j}\right]_{i,j=1}^{N},~P_{f} = P_{c} - P_{s}.
\end{equation}
We also indicate the channel state probability at time $k$ as
\begin{equation}\label{eq:pik}
     \mathbb{P}(\theta_{k} = s_i) = \pi_{i}(k),~\pi(k)\triangleq\left[\pi_{i}(k)\right]_{i=1}^{N},
\end{equation}
a column vector, so that $\pi(t)=\pi(0)P_c^t$ $\forall t\in\mathbb{Z}^{0+}$.
Inspired by Costa et al., we use the detector with an emission probability matrix (EPM) $P_{e}^{} \triangleq \left[\alpha_{i\mu}\right]_{i,\mu=1}^{N,M}$, where
\begin{equation}\label{eq:alpha}
   \mathbb{P}(\hat{\theta}_{k} = \hat{s}_{\mu} \mid \theta_{k} = s_i) = \alpha_{i\mu} \geq 0,~ \sum_{i=1}^N \alpha_{i\mu}= 1.
\end{equation}
We then define the following diagonal matrices.
\begin{equation}\label{eq:epm}
     P_{\pi}^{}(k) = \bigoplus_{i=1}^{N} \pi_{i}(k),~P_{\alpha}^{}(\mu) = \bigoplus_{i=1}^{N} \alpha_{i\mu}
\end{equation}
We count the consecutive control message dropouts observed by a controller at time step $k$ in a stochastic variable $\mathit{\Delta}_{k}$.
\begin{equation}\label{eq:z-k}
    \mathit{\Delta}_{k}=(1-\delta_{k-1})(\mathit{\Delta}_{k-1}+1).
\end{equation}
\begin{equation}\label{eq:zl}
    \mathit{\Delta}_{k}=\ell\Leftrightarrow \delta_{k-1-\ell}=1 \land 
	\delta_{k-t}=0 ~ \forall t\in\mathbb{Z}^{+} : t\leq \ell.
\end{equation}
Notably, if $\ell\!=\!0$, then $\left\{t\right\}_{t=1}^{0} \!=\! \emptyset$, meaning that \eqref{eq:zl} becomes $\mathit{\Delta}_{k}\!=\!0\Leftrightarrow \delta_{k-1}\!=\!1$.
Denote by $\mathcal{T}$ a set of time instances in which actuators successfully receive the controller's messages: %. Formally, 
\begin{subequations}\label{eq:tau} 
\begin{equation}\label{eq:calT}
    \mathcal{T} \triangleq \left\{ k : \delta_k = 1 \right\}_{k\in\mathbb{Z}^{0+}} = \{ \tau_{(m)}\}_{m\in\mathbb{Z}^{0+}}.
\end{equation}
From \eqref{eq:z-k}, \eqref{eq:zl}, and \eqref{eq:calT}, for all $m\in\mathbb{Z}^{0+}$,
\begin{equation}\label{eq:deltatau}
    \tau_{(m)} \in \mathcal{T}  \Rightarrow \delta_{\tau_{(m)}} = 1 \Rightarrow \mathit{\Delta}_{\tau_{(m)}+1} = 0,
\end{equation}
\begin{equation}\label{eq:tau+}
    \tau_{(m+1)} = \tau_{(m)} + 1 + \mathit{\Delta}_{\tau_{(m+1)}}.
\end{equation}
\end{subequations}

For the notational convenience, for any $k,n\in\mathbb{Z}^{0+}$, let
\begin{equation}\label{eq:calPsi}
W_{\!(k,n)} \!\triangleq\! \sum\nolimits_{j=0}^{n}A^{n-j} w_{k+j}^{}.
\end{equation}
To provide the stochastic characterization of the system dynamics, we group the current duration of a packet error burst and the last detected wireless channel state in one augmented discrete state:
$\eta_{k}^{} \!\triangleq\! (\mathit{\Delta}_{k},\hat{\theta}_{k-1})$. Then, from \eqref{eq:tau},
\begin{subequations}\label{eq:etatau}
\begin{equation}\label{eq:etataum}
    \eta_{\tau_{(m)}}^{} = (\mathit{\Delta}_{\tau_{(m)}},\hat{\theta}_{\tau_{(m)}-1}),
\end{equation}
\begin{equation}\label{eq:etataump}
    \eta_{\tau_{(m+1)}}^{} = (\mathit{\Delta}_{\tau_{(m+1)}},\hat{\theta}_{\tau_{(m)}+\mathit{\Delta}_{\tau_{(m+1)}}}),
\end{equation}
\end{subequations}
\newpage
\begin{figure*}[ht]
\raggedright
where $\mathit{\Delta}_{\tau_{(m+1)}}$ indicates the time interval the transmitted control input may remain active. From the controller's perspective, $\eta_{\tau_{(m)}}^{}$ is known, while $\eta_{\tau_{(m+1)}}^{}$ is a random variable.
Let $e_i$ and $e_j$ denote the $i$th and $j$th column vectors of the standard basis of $\mathbb{R}^{N}$ (all their components are zero except the $i$th for $e_i$ and $j$th for $e_j$, and the nonzero element equals one), and $\mathbf{1}$ indicates the column vector of length $N$ with all components equal one.
\begin{align}\label{eq:proof0}
\begin{aligned}
&\mathbb{P}(\eta_{\tau_{(m+1)}}^{} = (n,\hat{s}_{\nu}) \mid \eta_{\tau_{(m)}}^{} = (\ell,\hat{s}_{\mu})) = \\
&\labelrel={proof0:1}
\mathbb{P}(\eta_{\tau_{(m+1)}}^{} = (n,\hat{s}_{\nu}), \tau_{(m+1)} \in \mathcal{T} \mid \eta_{\tau_{(m)}}^{} = (\ell,\hat{s}_{\mu}), \tau_{(m)} \in \mathcal{T}) = \\
&\labelrel={proof0:2}
\mathbb{P}(\mathit{\Delta}_{\tau_{(m+1)}}=n,\hat{\theta}_{\tau_{(m)}+\mathit{\Delta}_{\tau_{(m+1)}}} = \hat{s}_{\nu}, \tau_{(m+1)} \in \mathcal{T} \mid \mathit{\Delta}_{\tau_{(m)}}=\ell,\hat{\theta}_{\tau_{(m)}-1} = \hat{s}_{\mu},\tau_{(m)} \in \mathcal{T}) = \\
&\labelrel={proof0:3}
\mathbb{P}(\mathit{\Delta}_{\tau_{(m+1)}}=n,\hat{\theta}_{\tau_{(m)}+n} = \hat{s}_{\nu}, \delta_{\tau_{(m+1)}}=1 \mid \mathit{\Delta}_{\tau_{(m)}}=\ell,\hat{\theta}_{\tau_{(m)}-1} = \hat{s}_{\mu}, \delta_{\tau_{(m)}}=1) = \\
&\labelrel={proof0:4}
\mathbb{P}(\delta_{\tau_{(m)}}=1, \delta_{\tau_{(m)}+t_{1}}=0 ~ \forall t_{1}\in\mathbb{Z}^{+} : t_{1}\leq n ,\hat{\theta}_{\tau_{(m)}+n} = \hat{s}_{\nu}, \delta_{\tau_{(m)+1+n}}=1 \mid \\
& \qquad~~ \mathit{\Delta}_{\tau_{(m)}}=\ell,\hat{\theta}_{\tau_{(m)}-1} = \hat{s}_{\mu}, \delta_{\tau_{(m)}}=1) = \\
&\labelrel={proof0:5}
\mathbb{P}(\delta_{\tau_{(m)}}=1, \delta_{\tau_{(m)}+t_{1}}=0 ~ \forall t_{1}\in\mathbb{Z}^{+} : t_{1}\leq n ,\hat{\theta}_{\tau_{(m)}+n} = \hat{s}_{\nu}, \delta_{\tau_{(m)+1+n}}=1 \mid \\
& \qquad~~ \delta_{\tau_{(m)}-\ell-1}=1, \delta_{\tau_{(m)}-t_{2}}=0 ~ \forall t_{2}\in\mathbb{Z}^{+} : t_{2}\leq \ell, \hat{\theta}_{\tau_{(m)}-1} = \hat{s}_{\mu}, \delta_{\tau_{(m)}}=1) = \\
&\labelrel={proof0:6}
\mathbb{P}(\delta_{\tau_{(m)}+t_{1}}=0 ~ \forall t_{1}\in\mathbb{Z}^{+} : t_{1}\leq n,\hat{\theta}_{\tau_{(m)}+n} = \hat{s}_{\nu}, \delta_{\tau_{(m)+1+n}}=1 \mid \\
& \qquad~~ \delta_{\tau_{(m)}-\ell-1}=1, \delta_{\tau_{(m)}-t_{2}}=0 ~ \forall t_{2}\in\mathbb{Z}^{+} : t_{2}\leq \ell, \hat{\theta}_{\tau_{(m)}-1} = \hat{s}_{\mu}, \delta_{\tau_{(m)}}=1) = \\
&\labelrel={proof0:7}
\mathbb{P}(\delta_{\tau_{(m)}+t_{1}}=0 ~ \forall t_{1}\in\mathbb{Z}^{+} : t_{1}\leq n,\hat{\theta}_{\tau_{(m)}+n} = \hat{s}_{\nu}, \delta_{\tau_{(m)+1+n}}=1 \mid \hat{\theta}_{\tau_{(m)}-1} = \hat{s}_{\mu}, \delta_{\tau_{(m)}}=1) = \\
&\labelrel={proof0:8}
\frac{\mathbb{P}(\delta_{\tau_{(m)+1+n}}=1, \hat{\theta}_{\tau_{(m)}+n} = \hat{s}_{\nu}, \delta_{\tau_{(m)}+t_{1}}=0 ~ \forall t_{1}\in\mathbb{Z}^{+} : t_{1}\leq n, \hat{\theta}_{\tau_{(m)}-1} = \hat{s}_{\mu}, \delta_{\tau_{(m)}}=1)}{\mathbb{P}(\hat{\theta}_{\tau_{(m)}-1} = \hat{s}_{\mu}, \delta_{\tau_{(m)}}=1)}.%  = \\
% &\labelrel={proof0:9}
% \frac{\mathbb{P}(\delta_{\tau_{(m)+1+n}}=1, \hat{\theta}_{\tau_{(m)}+n} = \hat{s}_{\nu}, \delta_{\tau_{(m)}+t_{1}}=0 ~ \forall t_{1}\in\mathbb{Z}^{+} : t_{1}\leq n, \delta_{\tau_{(m)}}=1 \mid \hat{\theta}_{\tau_{(m)}-1} = \hat{s}_{\mu})\mathbb{P}(\hat{\theta}_{\tau_{(m)}-1} = \hat{s}_{\mu})}{\mathbb{P}(\delta_{\tau_{(m)}}=1 \mid \hat{\theta}_{\tau_{(m)}-1} = \hat{s}_{\mu})\mathbb{P}(\hat{\theta}_{\tau_{(m)}-1} = \hat{s}_{\mu})}  = \\
% &\labelrel={proof0:10}
% \frac{e_i^{\top} P_{\!s}(\tau_{(m)}) \prod\nolimits_{t=1}^{n} P_{\!f}(\tau_{(m)}+t) e_{j}^{}e_{j}^{\top} P_{\!s}(\tau_{(m+1)}) \mathbf{1}}{e_i^{\top} P_{\!s}(\tau_{(m)})\mathbf{1}}, % \triangleq \zeta_{(i,n,j)}.
\end{aligned}
\end{align}
where commas indicate the intersections of events, and a colon specifies a constraint (such that);
\begin{itemize}
\item the equality~\eqref{proof0:1} is due to \eqref{eq:calT}, i.e., $\forall m \in \mathbb{Z}^{0+}$, it holds that $\tau_{m}\in\mathcal{T}$, with $\mathcal{T}\triangleq \left\{ k : \delta_k = 1 \right\}_{k\in \mathbb{Z}^{0+}}$;
\item \eqref{proof0:2} relies on \eqref{eq:etatau}, i.e., $ \eta_{\tau_{(m+1)}}^{} = (\mathit{\Delta}_{\tau_{(m+1)}},\hat{\theta}_{\tau_{(m)}+\mathit{\Delta}_{\tau_{(m+1)}}})$ and $ \eta_{\tau_{(m)}}^{} = (\mathit{\Delta}_{\tau_{(m)}},\hat{\theta}_{\tau_{(m)}-1})$;
\item \eqref{proof0:3} substitutes $\mathit{\Delta}_{\tau_{(m+1)}}$ with $n$ in $\hat{\theta}_{\tau_{(m)}+\mathit{\Delta}_{\tau_{(m+1)}}}$ and applies \eqref{eq:deltatau};
\item \eqref{proof0:4} uses \eqref{eq:tau+} and applies \eqref{eq:zl} for $k=\tau_{(m+1)}$;
\item \eqref{proof0:5} applies \eqref{eq:zl} for $k=\tau_{(m)}$;
\item \eqref{proof0:6} uses the chain rule of the probability and states that for three events $A$, $B$ and $C$ it holds that 
	\begin{equation}\label{eq:chain-rule-special}
		\mathbb{P}(A \cap B \mid B \cap C ) = \mathbb{P}(A \mid B \cap B \cap C ) \, 
		\mathbb{P}(B \mid B \cap C) = \mathbb{P}(A \mid B \cap C)
	\end{equation}
	since the intersection of a set with itself is the set itself, and applying the definition of the conditional probability, i.e.,
	\begin{equation*}
		B\cap B = B \qquad 
		\mathbb{P}(A\mid B)=\frac{\mathbb{P}(A\cap B)}{\mathbb{P}(B)} \qquad
		\Rightarrow \qquad \mathbb{P}(B\mid B\cap C)=
		\frac{\mathbb{P}(B\cap B\cap C)}{\mathbb{P}(B\cap C)} =
		\frac{\mathbb{P}(B\cap C)}{\mathbb{P}(B\cap C)} = 1;
	\end{equation*}
\item \eqref{proof0:7} is an immediate consequence of the independence of both $\delta_{k}$ and $\hat{\theta}_{k}$ from $\delta_{k-t_{3}}$, for all $t_{3}\geq 1$;
\item \eqref{proof0:8} uses the conditional probability definition and the commutative property of the set intersection.%;
% \item \eqref{proof0:9} relies on the chain rule of probability;
% \item \eqref{proof0:10} cancels the common factor $\mathbb{P}(\theta_{\tau_{(m)}-1} = s_i)$, applies the law of total probability, the independence of both $\delta_{k}$ and $\theta_{k}$ from $\delta_{k-t_{3}}$, the Markov property, the commutative property of the multiplication and the addition of real numbers, and the definitions of $P_{\!s}(k)$ and $P_{\!f}(k)$ in (5).
\end{itemize}
\end{figure*}
\begin{figure*}[ht]
\raggedright
The denominator of \eqref{eq:proof0} has the following expression.
\begin{align}\label{eq:proof1}
\begin{aligned}
&\mathbb{P}(\hat{\theta}_{\tau_{(m)}-1} = \hat{s}_{\mu}, \delta_{\tau_{(m)}}=1) = \\
&\labelrel={proof1:1} \sum_{i=1}^{N}\sum_{j=1}^{N} \,\mathbb{P}(\delta_{\tau_{(m)}}=1,\theta_{\tau_{(m)}}=s_j,\hat{\theta}_{\tau_{(m)}-1} = \hat{s}_{\mu},\theta_{\tau_{(m)}-1}=s_i ) =\\
&\labelrel={proof1:2} \sum_{i=1}^{N}\sum_{j=1}^{N}\,\mathbb{P}(\delta_{\tau_{(m)}}=1 \mid \theta_{\tau_{(m)}}=s_j,\hat{\theta}_{\tau_{(m)}-1} = \hat{s}_{\mu},\theta_{\tau_{(m)}-1}=s_i ) \cdot
\mathbb{P}(\theta_{\tau_{(m)}}=s_j \mid \hat{\theta}_{\tau_{(m)}-1} = \hat{s}_{\mu},\theta_{\tau_{(m)}-1}=s_i ) \cdot \\
&\qquad\qquad\quad\! \mathbb{P}(\hat{\theta}_{\tau_{(m)}-1} = \hat{s}_{\mu} \mid \theta_{\tau_{(m)}-1}=s_i )\cdot\mathbb{P}(\theta_{\tau_{(m)}-1}=s_i ) = \\
&\labelrel={proof1:3} \sum_{i=1}^{N}\sum_{j=1}^{N}\,\mathbb{P}(\delta_{\tau_{(m)}}=1 \mid \theta_{\tau_{(m)}}=s_j) \cdot
\mathbb{P}(\theta_{\tau_{(m)}}=s_j \mid \hat{\theta}_{\tau_{(m)}-1} = \hat{s}_{\mu},\theta_{\tau_{(m)}-1}=s_i ) \cdot \\
&\qquad\qquad\quad\! \mathbb{P}(\hat{\theta}_{\tau_{(m)}-1} = \hat{s}_{\mu} \mid \theta_{\tau_{(m)}-1}=s_i )\cdot\mathbb{P}(\theta_{\tau_{(m)}-1}=s_i ) = \\
&\labelrel={proof1:4} \sum_{i=1}^{N}\sum_{j=1}^{N}\,\hat{\delta}_{j} \cdot
\mathbb{P}(\theta_{\tau_{(m)}}=s_j \mid \hat{\theta}_{\tau_{(m)}-1} = \hat{s}_{\mu},\theta_{\tau_{(m)}-1}=s_i ) \cdot \alpha_{i\mu}\cdot\mathbb{P}(\theta_{\tau_{(m)}-1}=s_i ) = \\
&\labelrel={proof1:5} \sum_{i=1}^{N}\sum_{j=1}^{N}\,\hat{\delta}_{j} \cdot
\mathbb{P}(\theta_{\tau_{(m)}}=s_j \mid \theta_{\tau_{(m)}-1}=s_i ) \cdot \alpha_{i\mu}\cdot\mathbb{P}(\theta_{\tau_{(m)}-1}=s_i ) = \\
&\labelrel={proof1:6} \sum_{i=1}^{N}\sum_{j=1}^{N}\,\alpha_{i\mu}\pi_{i}(\tau{(m)}-1)p_{ij}\hat{\delta}_{j} = \\
&\labelrel={proof1:7} \hat{e}_{\mu}^{\top} P_{e}^{\top} P_{\pi}(\tau{(m)}-1) P_{s}\mathbf{1} .
\end{aligned}
\end{align}
where the equalities hold for the following reasons:
\begin{itemize}
    \item \eqref{proof1:1} is due to the law of total probability and commutative property of set intersection;
    \item \eqref{proof1:2} follows from the chain rule of probability;
    \item \eqref{proof1:3} is a consequence of the independence of $\delta_{\tau_{(m)}}$ from $\theta_{\tau_{(m)}-t_{1}}$ and $\hat{\theta}_{\tau_{(m)}-t_{1}}$, for all $t_{1}\geq 1$;
    \item \eqref{proof1:4} relies on \eqref{eq:p-delta} and \eqref{eq:alpha};
    \item \eqref{proof1:5} uses the independence of $\theta_{\tau_{(m)}}$ from $\hat{\theta}_{\tau_{(m)}-1}$;
    \item \eqref{proof1:6} is obtained from \eqref{eq:p-ij}, \eqref{eq:pik}, and the commutative property of the scalar product;
    \item \eqref{proof1:7} applies \eqref{eq:Ps}, \eqref{eq:epm}, and the definition of the emission probability matrix. $\hat{e}_{\mu}$ denotes the $\mu$th column vectors of the standard basis of $\mathbb{R}^{M}$ (all its components are zero except the $\mu$th, and the nonzero element equals one).
\end{itemize}
\end{figure*}
\begin{figure*}[ht]
\raggedright
The nominator of \eqref{eq:proof0} has the following expression.
\begin{align}\label{eq:proof2}
\begin{aligned}
&\mathbb{P}(\delta_{\tau_{(m)+1+n}}=1, \hat{\theta}_{\tau_{(m)}+n} = \hat{s}_{\nu}, \delta_{\tau_{(m)}+t_{1}}=0 ~ \forall t_{1}\in\mathbb{Z}^{+} : t_{1}\leq n, \hat{\theta}_{\tau_{(m)}-1} = \hat{s}_{\mu}, \delta_{\tau_{(m)}}=1) = \\
%%%%%%%%%%%%%%%%%%%%%%%%%%%%%%%%%%%%%%%%%%%%%%%%
&\labelrel={proof2:1} \sum_{i=1}^{N}\sum_{j=1}^{N}\sum_{i_1=1}^{N}\cdots\sum_{i_n=1}^{N}\sum_{i_{n+1}=1}^{N} \,\mathbb{P}(\delta_{\tau_{(m)+1+n}}=1, \theta_{\tau_{(m)}+1+n}=s_{i_{n+1}}, \hat{\theta}_{\tau_{(m)}+n} = \hat{s}_{\nu}, \delta_{\tau_{(m)}+n}=0, \theta_{\tau_{(m)}+n}=s_{i_{n}},\dots, \\
& \qquad\qquad\quad\! 
\delta_{\tau_{(m)}+1}=0,\theta_{\tau_{(m)}+1}=s_{i_{1}},\delta_{\tau_{(m)}}=1,\theta_{\tau_{(m)}}=s_j,\hat{\theta}_{\tau_{(m)}-1} = \hat{s}_{\mu},\theta_{\tau_{(m)}-1}=s_i) = \\
%%%%%%%%%%%%%%%%%%%%%%%%%%%%%%%%%%%%%%%%%%%%%%%%
&\labelrel={proof2:2} \sum_{i=1}^{N}\sum_{j=1}^{N}\sum_{i_1=1}^{N}\cdots\sum_{i_n=1}^{N}\sum_{i_{n+1}=1}^{N} \,\mathbb{P}(\delta_{\tau_{(m)+1+n}}=1 \mid \theta_{\tau_{(m)}+1+n}=s_{i_{n+1}}, \hat{\theta}_{\tau_{(m)}+n} = \hat{s}_{\nu}, \delta_{\tau_{(m)}+n}=0, \theta_{\tau_{(m)}+n}=s_{i_{n}},\dots, \\
& \qquad\qquad\quad\! \delta_{\tau_{(m)}+1}=0,\theta_{\tau_{(m)}+1}=s_{i_{1}},\delta_{\tau_{(m)}}=1,\theta_{\tau_{(m)}}=s_j,\hat{\theta}_{\tau_{(m)}-1} = \hat{s}_{\mu},\theta_{\tau_{(m)}-1}=s_i) \cdot \\
& \qquad\quad~~~\! \mathbb{P}(\theta_{\tau_{(m)}+1+n}=s_{i_{n+1}} \mid \hat{\theta}_{\tau_{(m)}+n} = \hat{s}_{\nu}, \delta_{\tau_{(m)}+n}=0, \theta_{\tau_{(m)}+n}=s_{i_{n}},\dots,\delta_{\tau_{(m)}+1}=0,\theta_{\tau_{(m)}+1}=s_{i_{1}}, \\
& \qquad\qquad\quad\! \delta_{\tau_{(m)}}=1,\theta_{\tau_{(m)}}=s_j,\hat{\theta}_{\tau_{(m)}-1} = \hat{s}_{\mu},\theta_{\tau_{(m)}-1}=s_i)\cdot \\
& \qquad\quad~~~\! \mathbb{P}(\hat{\theta}_{\tau_{(m)}+n} = \hat{s}_{\nu} \mid \delta_{\tau_{(m)}+n}=0, \theta_{\tau_{(m)}+n}=s_{i_{n}},\dots,\delta_{\tau_{(m)}+1}=0,\theta_{\tau_{(m)}+1}=s_{i_{1}},\delta_{\tau_{(m)}}=1,\theta_{\tau_{(m)}}=s_j, \\
& \qquad\qquad\quad\! \hat{\theta}_{\tau_{(m)}-1} = \hat{s}_{\mu},\theta_{\tau_{(m)}-1}=s_i)\cdot \\
& \qquad\quad~~~\! \mathbb{P}(\delta_{\tau_{(m)}+n}=0 \mid \theta_{\tau_{(m)}+n}=s_{i_{n}},\dots,\delta_{\tau_{(m)}+1}=0,\theta_{\tau_{(m)}+1}=s_{i_{1}},\delta_{\tau_{(m)}}=1,\theta_{\tau_{(m)}}=s_j, \hat{\theta}_{\tau_{(m)}-1} = \hat{s}_{\mu},\\
& \qquad\qquad\quad\! \theta_{\tau_{(m)}-1}=s_i)\cdot \\
& \qquad\quad~~~\!
\mathbb{P}(\theta_{\tau_{(m)}+n}=s_{i_{n}}\mid\dots,\delta_{\tau_{(m)}+1}=0,\theta_{\tau_{(m)}+1}=s_{i_{1}},\delta_{\tau_{(m)}}=1,\theta_{\tau_{(m)}}=s_j, \hat{\theta}_{\tau_{(m)}-1} = \hat{s}_{\mu},\theta_{\tau_{(m)}-1}=s_i)\cdots \\
& \qquad\quad~~~\!
\mathbb{P}(\delta_{\tau_{(m)}+1}=0 \mid \theta_{\tau_{(m)}+1}=s_{i_{1}},\delta_{\tau_{(m)}}=1,\theta_{\tau_{(m)}}=s_j, \hat{\theta}_{\tau_{(m)}-1} = \hat{s}_{\mu},\theta_{\tau_{(m)}-1}=s_i)\cdot \\
& \qquad\quad~~~\!
\mathbb{P}(\theta_{\tau_{(m)}+1}=s_{i_{1}} \mid \delta_{\tau_{(m)}}=1,\theta_{\tau_{(m)}}=s_j, \hat{\theta}_{\tau_{(m)}-1} = \hat{s}_{\mu},\theta_{\tau_{(m)}-1}=s_i)\cdot \\
& \qquad\quad~~~\!
\mathbb{P}(\delta_{\tau_{(m)}}=1 \mid \theta_{\tau_{(m)}}=s_j, \hat{\theta}_{\tau_{(m)}-1} = \hat{s}_{\mu},\theta_{\tau_{(m)}-1}=s_i)\cdot \\
& \qquad\quad~~~\!
\mathbb{P}(\theta_{\tau_{(m)}}=s_j \mid \hat{\theta}_{\tau_{(m)}-1} = \hat{s}_{\mu},\theta_{\tau_{(m)}-1}=s_i)\cdot 
\mathbb{P}(\hat{\theta}_{\tau_{(m)}-1} = \hat{s}_{\mu}\mid\theta_{\tau_{(m)}-1}=s_i) \cdot 
\mathbb{P}(\theta_{\tau_{(m)}-1}=s_i)\\
%%%%%%%%%%%%%%%%%%%%%%%%%%%%%%%%%%%%%%%%%%%%%%%%
&\labelrel={proof2:3} \sum_{i=1}^{N}\sum_{j=1}^{N}\sum_{i_1=1}^{N}\cdots\sum_{i_n=1}^{N}\sum_{i_{n+1}=1}^{N} \,\mathbb{P}(\delta_{\tau_{(m)+1+n}}=1 \mid \theta_{\tau_{(m)}+1+n}=s_{i_{n+1}}) \cdot \\
& \qquad\quad~~~\! \mathbb{P}(\theta_{\tau_{(m)}+1+n}=s_{i_{n+1}} \mid \hat{\theta}_{\tau_{(m)}+n} = \hat{s}_{\nu}, \delta_{\tau_{(m)}+n}=0, \theta_{\tau_{(m)}+n}=s_{i_{n}},\dots,\delta_{\tau_{(m)}+1}=0,\theta_{\tau_{(m)}+1}=s_{i_{1}}, \\
& \qquad\qquad\quad\! \delta_{\tau_{(m)}}=1,\theta_{\tau_{(m)}}=s_j,\hat{\theta}_{\tau_{(m)}-1} = \hat{s}_{\mu},\theta_{\tau_{(m)}-1}=s_i)\cdot \\
& \qquad\quad~~~\! \mathbb{P}(\hat{\theta}_{\tau_{(m)}+n} = \hat{s}_{\nu} \mid \delta_{\tau_{(m)}+n}=0, \theta_{\tau_{(m)}+n}=s_{i_{n}},\dots,\delta_{\tau_{(m)}+1}=0,\theta_{\tau_{(m)}+1}=s_{i_{1}},\delta_{\tau_{(m)}}=1,\theta_{\tau_{(m)}}=s_j, \\
& \qquad\qquad\quad\! \hat{\theta}_{\tau_{(m)}-1} = \hat{s}_{\mu},\theta_{\tau_{(m)}-1}=s_i)\cdot \\
& \qquad\quad~~~\! \mathbb{P}(\delta_{\tau_{(m)}+n}=0 \mid \theta_{\tau_{(m)}+n}=s_{i_{n}})\cdot \\
& \qquad\quad~~~\!\mathbb{P}(\theta_{\tau_{(m)}+n}=s_{i_{n}}\mid\dots,\delta_{\tau_{(m)}+1}=0,\theta_{\tau_{(m)}+1}=s_{i_{1}},\delta_{\tau_{(m)}}=1,\theta_{\tau_{(m)}}=s_j, \hat{\theta}_{\tau_{(m)}-1} = \hat{s}_{\mu},\theta_{\tau_{(m)}-1}=s_i)\cdots \\
& \qquad\quad~~~\!
\mathbb{P}(\delta_{\tau_{(m)}+1}=0 \mid \theta_{\tau_{(m)}+1}=s_{i_{1}})\cdot \\
& \qquad\quad~~~\!
\mathbb{P}(\theta_{\tau_{(m)}+1}=s_{i_{1}} \mid \delta_{\tau_{(m)}}=1,\theta_{\tau_{(m)}}=s_j, \hat{\theta}_{\tau_{(m)}-1} = \hat{s}_{\mu},\theta_{\tau_{(m)}-1}=s_i)\cdot \\
& \qquad\quad~~~\!
\mathbb{P}(\delta_{\tau_{(m)}}=1 \mid \theta_{\tau_{(m)}}=s_j)\cdot \\
& \qquad\quad~~~\!
\mathbb{P}(\theta_{\tau_{(m)}}=s_j \mid \hat{\theta}_{\tau_{(m)}-1} = \hat{s}_{\mu},\theta_{\tau_{(m)}-1}=s_i)\cdot 
\mathbb{P}(\hat{\theta}_{\tau_{(m)}-1} = \hat{s}_{\mu}\mid\theta_{\tau_{(m)}-1}=s_i) \cdot 
\mathbb{P}(\theta_{\tau_{(m)}-1}=s_i),
%%%%%%%%%%%%%%%%%%%%%%%%%%%%%%%%%%%%%%%%%%%%%%%%
\end{aligned}
\end{align}
where 
\begin{itemize}
    \item \eqref{proof2:1} is due to the law of total probability and commutative property of set intersection;
    \item \eqref{proof2:2} follows from the chain rule of probability;
    \item \eqref{proof2:3} is a consequence of the independence of $\delta_{k}$ from $\delta_{k-t_{1}}$, $\theta_{k-t_{1}}$, and $\hat{\theta}_{k}$ for all $t_{1}\geq 1$.
\end{itemize}
\end{figure*}
\begin{figure*}[ht]
\raggedright
Further derivation of the nominator of \eqref{eq:proof0} is as follows.
\begin{align}\label{eq:proof3}
\begin{aligned}
&\eqref{eq:proof2} 
%%%%%%%%%%%%%%%%%%%%%%%%%%%%%%%%%%%%%%%%%%%%%%%%
\labelrel={proof3:1} \sum_{i=1}^{N}\sum_{j=1}^{N}\sum_{i_1=1}^{N}\cdots\sum_{i_n=1}^{N}\sum_{i_{n+1}=1}^{N} \,\mathbb{P}(\delta_{\tau_{(m)+1+n}}=1 \mid \theta_{\tau_{(m)}+1+n}=s_{i_{n+1}}) \cdot \\
& \qquad\quad~~~\! \mathbb{P}(\theta_{\tau_{(m)}+1+n}=s_{i_{n+1}} \mid \theta_{\tau_{(m)}+n}=s_{i_{n}},\dots,\theta_{\tau_{(m)}+1}=s_{i_{1}}, \theta_{\tau_{(m)}}=s_j,\theta_{\tau_{(m)}-1}=s_i)\cdot \\
& \qquad\quad~~~\! \mathbb{P}(\hat{\theta}_{\tau_{(m)}+n} = \hat{s}_{\nu} \mid \delta_{\tau_{(m)}+n}=0, \theta_{\tau_{(m)}+n}=s_{i_{n}},\dots,\delta_{\tau_{(m)}+1}=0,\theta_{\tau_{(m)}+1}=s_{i_{1}},\delta_{\tau_{(m)}}=1,\theta_{\tau_{(m)}}=s_j, \\
& \qquad\qquad\quad\! \hat{\theta}_{\tau_{(m)}-1} = \hat{s}_{\mu},\theta_{\tau_{(m)}-1}=s_i)\cdot \\
& \qquad\quad~~~\! \mathbb{P}(\delta_{\tau_{(m)}+n}=0 \mid \theta_{\tau_{(m)}+n}=s_{i_{n}})\cdot 
\mathbb{P}(\theta_{\tau_{(m)}+n}=s_{i_{n}}\mid\dots,\theta_{\tau_{(m)}+1}=s_{i_{1}},\theta_{\tau_{(m)}}=s_j, \theta_{\tau_{(m)}-1}=s_i)\cdots \\
& \qquad\quad~~~\!
\mathbb{P}(\delta_{\tau_{(m)}+1}=0 \mid \theta_{\tau_{(m)}+1}=s_{i_{1}})\cdot 
\mathbb{P}(\theta_{\tau_{(m)}+1}=s_{i_{1}} \mid \theta_{\tau_{(m)}}=s_j,\theta_{\tau_{(m)}-1}=s_i)\cdot \\
& \qquad\quad~~~\!
\mathbb{P}(\delta_{\tau_{(m)}}=1 \mid \theta_{\tau_{(m)}}=s_j)\cdot
\mathbb{P}(\theta_{\tau_{(m)}}=s_j \mid \theta_{\tau_{(m)}-1}=s_i)\cdot 
\mathbb{P}(\hat{\theta}_{\tau_{(m)}-1} = \hat{s}_{\mu}\mid\theta_{\tau_{(m)}-1}=s_i) \cdot 
\mathbb{P}(\theta_{\tau_{(m)}-1}=s_i) = \\
%%%%%%%%%%%%%%%%%%%%%%%%%%%%%%%%%%%%%%%%%%%%%%%%
& \labelrel={proof3:2} \sum_{i=1}^{N}\sum_{j=1}^{N}\sum_{i_1=1}^{N}\cdots\sum_{i_n=1}^{N}\sum_{i_{n+1}=1}^{N} \,\mathbb{P}(\delta_{\tau_{(m)+1+n}}=1 \mid \theta_{\tau_{(m)}+1+n}=s_{i_{n+1}}) \cdot \\
& \qquad\quad~~~\! \mathbb{P}(\theta_{\tau_{(m)}+1+n}=s_{i_{n+1}} \mid \theta_{\tau_{(m)}+n}=s_{i_{n}},\dots,\theta_{\tau_{(m)}+1}=s_{i_{1}}, \theta_{\tau_{(m)}}=s_j,\theta_{\tau_{(m)}-1}=s_i)\cdot \\
& \qquad\quad~~~\! \mathbb{P}(\hat{\theta}_{\tau_{(m)}+n} = \hat{s}_{\nu} \mid \theta_{\tau_{(m)}+n}=s_{i_{n}})\cdot \\
& \qquad\quad~~~\! \mathbb{P}(\delta_{\tau_{(m)}+n}=0 \mid \theta_{\tau_{(m)}+n}=s_{i_{n}})\cdot 
\mathbb{P}(\theta_{\tau_{(m)}+n}=s_{i_{n}}\mid\dots,\theta_{\tau_{(m)}+1}=s_{i_{1}},\theta_{\tau_{(m)}}=s_j, \theta_{\tau_{(m)}-1}=s_i)\cdots \\
& \qquad\quad~~~\!
\mathbb{P}(\delta_{\tau_{(m)}+1}=0 \mid \theta_{\tau_{(m)}+1}=s_{i_{1}})\cdot 
\mathbb{P}(\theta_{\tau_{(m)}+1}=s_{i_{1}} \mid \theta_{\tau_{(m)}}=s_j,\theta_{\tau_{(m)}-1}=s_i)\cdot \\
& \qquad\quad~~~\!
\mathbb{P}(\delta_{\tau_{(m)}}=1 \mid \theta_{\tau_{(m)}}=s_j)\cdot
\mathbb{P}(\theta_{\tau_{(m)}}=s_j \mid \theta_{\tau_{(m)}-1}=s_i)\cdot 
\mathbb{P}(\hat{\theta}_{\tau_{(m)}-1} = \hat{s}_{\mu}\mid\theta_{\tau_{(m)}-1}=s_i) \cdot 
\mathbb{P}(\theta_{\tau_{(m)}-1}=s_i) = \\
%%%%%%%%%%%%%%%%%%%%%%%%%%%%%%%%%%%%%%%%%%%%%%%%
& \labelrel={proof3:3} \sum_{i=1}^{N}\sum_{j=1}^{N}\sum_{i_1=1}^{N}\cdots\sum_{i_n=1}^{N}\sum_{i_{n+1}=1}^{N} \,\hat{\delta}_{i_{n+1}} \mathbb{P}(\theta_{\tau_{(m)}+1+n}=s_{i_{n+1}} \mid \theta_{\tau_{(m)}+n}=s_{i_{n}}) \, \alpha_{i_{n}\nu} \, (1-\hat{\delta}_{i_n}) \cdot \\
& \qquad\quad~~~\! \mathbb{P}(\theta_{\tau_{(m)}+n}=s_{i_{n}}\mid \theta_{\tau_{(m)}+n-1}=s_{i_{n-1}})\cdots (1-\hat{\delta}_{i_1})
\mathbb{P}(\theta_{\tau_{(m)}+1}=s_{i_{1}} \mid \theta_{\tau_{(m)}}=s_j)
\hat{\delta}_{j}\,p_{ij}\alpha_{i\mu} \mathbb{P}(\theta_{\tau_{(m)}-1}=s_i) = \\
%%%%%%%%%%%%%%%%%%%%%%%%%%%%%%%%%%%%%%%%%%%%%%%%
& \labelrel={proof3:4} \sum_{i=1}^{N}\sum_{j=1}^{N}\sum_{i_1=1}^{N}\cdots\sum_{i_n=1}^{N}\sum_{i_{n+1}=1}^{N} \hat{\delta}_{i_{n+1}} p_{i_{n} i_{n+1}} \, \alpha_{i_{n}\nu} (1-\hat{\delta}_{i_n}) p_{i_{n-1} i_{n}} \cdots (1-\hat{\delta}_{i_1}) p_{j i_{1}} \hat{\delta}_{j} p_{ij} \alpha_{i\mu} \pi_{i}(\tau_{(m)}-1) = \\
%%%%%%%%%%%%%%%%%%%%%%%%%%%%%%%%%%%%%%%%%%%%%%%%
& \labelrel={proof3:5} \sum_{i=1}^{N}\sum_{j=1}^{N}\sum_{i_1=1}^{N}\cdots\sum_{i_n=1}^{N}\sum_{i_{n+1}=1}^{N} 
\alpha_{i\mu} \pi_{i}(\tau_{(m)}-1) p_{ij} \hat{\delta}_{j} \cdot p_{j i_{1}} (1-\hat{\delta}_{i_1}) \cdots
p_{i_{n-1} i_{n}} (1-\hat{\delta}_{i_n}) \alpha_{i_{n}\nu} p_{i_{n} i_{n+1}} \hat{\delta}_{i_{n+1}}  = \\
%%%%%%%%%%%%%%%%%%%%%%%%%%%%%%%%%%%%%%%%%%%%%%%%
& \labelrel={proof3:6} \hat{e}_{\mu}^{\top} P_{e}^{\top} P_{\pi}(\tau{(m)}-1) P_{s} P_{f}^{n} P_{\alpha}(\nu) P_{s} \mathbf{1}, \\
%%%%%%%%%%%%%%%%%%%%%%%%%%%%%%%%%%%%%%%%%%%%%%%%
\end{aligned}
\end{align}
where 
\begin{itemize}
    \item \eqref{proof3:1} relies on the independence of $\theta_{k}$ from $\hat{\theta}_{k-t_2}$ and $\delta_{k-t_2}$ for all $t_2\geq 0$;
    \item \eqref{proof3:2} applies the independence of $\hat{\theta}_{k}$ from $\theta_{k-t_3}$, $\hat{\theta}_{k-t_3}$, and $\delta_{k-t_4}$ for all $t_3\geq 1$ and $t_4\geq 0$;
    \item \eqref{proof3:3} uses the Markov property, \eqref{eq:p-delta}, and \eqref{eq:alpha};
    \item \eqref{proof3:4} is due to \eqref{eq:p-ij} and \eqref{eq:pik};
    \item \eqref{proof3:5} applies the commutative property of the scalar product;
    \item \eqref{proof3:6} applies \eqref{eq:Ps}, \eqref{eq:epm}, and the definition of the emission probability matrix.
\end{itemize}
\end{figure*}
\begin{figure*}[ht]
\raggedright
From \eqref{eq:proof1} and \eqref{eq:proof3}, we have that \eqref{eq:proof0} becomes
\begin{equation}\label{eq:prooff}
\mathbb{P}(\eta_{\tau_{(m+1)}}^{} = (n,\hat{s}_{\nu}) \mid \eta_{\tau_{(m)}}^{} = (\ell,\hat{s}_{\mu})) =
    \frac{\hat{e}_{\mu}^{\top} P_{e}^{\top} P_{\pi}(\tau{(m)}-1) P_{s} P_{f}^{n} P_{\alpha}(\nu) P_{s} \mathbf{1}}{\hat{e}_{\mu}^{\top} P_{e}^{\top} P_{\pi}(\tau{(m)}-1) P_{s}\mathbf{1}} \triangleq \zeta_{(\mu,n,\nu)}^{\tau_{(m)}}.
\end{equation}
\end{figure*}
\begin{figure*}[ht]
\raggedright
Notice that under the perfect channel state estimation $M=N$ and $P_e = I_{N}$ so that \eqref{eq:prooff} becomes
\begin{align}
\mathbb{P}(\eta&_{\tau_{(m+1)}}^{} = (n,\hat{s}_{\nu}=s_j) \mid \eta_{\tau_{(m)}}^{} = (\ell,\hat{s}_{\mu}=s_i)) =
    % \frac{\hat{e}_{\mu}^{\top} P_{e}^{\top} P_{\pi}(\tau{(m)}-1) P_{s} P_{f}^{n} P_{\alpha}(\nu) P_{s} \mathbf{1}}{\hat{e}_{\mu}^{\top} P_{e}^{\top} P_{\pi}(\tau{(m)}-1) P_{s}\mathbf{1}}=\\
    % &
    \frac{\pi_{i}(\tau_{(m)}-1)e_{i}^{\top} P_{s} P_{f}^{n} e_{j}e_{j}^{\top} P_{s} \mathbf{1}}{\pi_{i}(\tau_{(m)}-1)e_{i}^{\top} P_{s}\mathbf{1}} = \frac{e_{i}^{\top} P_{s} P_{f}^{n} e_{j}e_{j}^{\top} P_{s} \mathbf{1}}{e_{i}^{\top} P_{s}\mathbf{1}}.
\end{align}
since $\hat{e}_{\mu}^{\top} P_{e}^{\top} = e_{i}^{\top} I_{N} = e_{i}^{\top}$, $e_{i}^{\top} P_{\pi}(\tau{(m)}-1)=e_{i}^{\top}\pi_{i}(\tau_{(m)}-1)=\pi_{i}(\tau_{(m)}-1)e_{i}^{\top}$, and $P_{\alpha}(\nu=j)=e_{j}e_{j}^{\top}$.
\end{figure*}
\begin{figure*}[ht]
\raggedright
For notational convenience, 
\begin{equation}
    q_{\mu n}^{\tau_{(m)}} \triangleq \sum_{\nu=1}^{M} \zeta_{(\mu,n,\nu)}^{\tau_{(m)}}.
\end{equation}
\end{figure*}

% \begin{figure*}[ht]
% \raggedright
% Notice that 
% \begin{equation*}
% \begin{bmatrix}
%     p_{11}(1-\hat{\delta}_{1}) & p_{12}(1-\hat{\delta}_{2}) & p_{13}(1-\hat{\delta}_{3}) \\
%     p_{21}(1-\hat{\delta}_{1}) & p_{22}(1-\hat{\delta}_{2}) & p_{23}(1-\hat{\delta}_{3}) \\
%     p_{31}(1-\hat{\delta}_{1}) & p_{32}(1-\hat{\delta}_{2}) & p_{33}(1-\hat{\delta}_{3}) 
% \end{bmatrix}
% \begin{bmatrix}
%     \alpha_{11} & 0 & 0 \\
%     0 & \alpha_{21} & 0 \\
%     0 & 0 & \alpha_{31} \\
% \end{bmatrix}=
% \begin{bmatrix}
%     p_{11}(1-\hat{\delta}_{1})\alpha_{11} & p_{12}(1-\hat{\delta}_{2})\alpha_{21} & p_{13}(1-\hat{\delta}_{3})\alpha_{31} \\
%     p_{21}(1-\hat{\delta}_{1})\alpha_{11} & p_{22}(1-\hat{\delta}_{2})\alpha_{21} & p_{23}(1-\hat{\delta}_{3})\alpha_{31} \\
%     p_{31}(1-\hat{\delta}_{1})\alpha_{11} & p_{32}(1-\hat{\delta}_{2})\alpha_{21} & p_{33}(1-\hat{\delta}_{3})\alpha_{31} 
% \end{bmatrix}
% \end{equation*}
% \begin{equation*}
% \begin{bmatrix}
%     p_{11}\hat{\delta}_{1}  & p_{12}\hat{\delta}_{2}  & p_{13}\hat{\delta}_{3}  \\
%     p_{21}\hat{\delta}_{1}  & p_{22}\hat{\delta}_{2}  & p_{23}\hat{\delta}_{3}  \\
%     p_{31}\hat{\delta}_{1}  & p_{32}\hat{\delta}_{2}  & p_{33}\hat{\delta}_{3}  
% \end{bmatrix}
% \begin{bmatrix}
%     1 \\ 1 \\ 1
% \end{bmatrix} = 
% \begin{bmatrix}
%     p_{11}\hat{\delta}_{1}  + p_{12}\hat{\delta}_{2}  + p_{13}\hat{\delta}_{3} \\
%     p_{21}\hat{\delta}_{1}  + p_{22}\hat{\delta}_{2}  + p_{23}\hat{\delta}_{3} \\
%     p_{31}\hat{\delta}_{1}  + p_{32}\hat{\delta}_{2}  + p_{33}\hat{\delta}_{3} 
% \end{bmatrix}
% \end{equation*}
% \begin{equation*}
%     \begin{bmatrix}
%     \alpha_{11} & 0 & 0 \\
%     0 & \alpha_{21} & 0 \\
%     0 & 0 & \alpha_{31} \\
% \end{bmatrix}
% \begin{bmatrix}
%     p_{11}\hat{\delta}_{1}  & p_{12}\hat{\delta}_{2}  & p_{13}\hat{\delta}_{3}  \\
%     p_{21}\hat{\delta}_{1}  & p_{22}\hat{\delta}_{2}  & p_{23}\hat{\delta}_{3}  \\
%     p_{31}\hat{\delta}_{1}  & p_{32}\hat{\delta}_{2}  & p_{33}\hat{\delta}_{3}  
% \end{bmatrix}=
% \begin{bmatrix}
%     \alpha_{11}p_{11}\hat{\delta}_{1}  & \alpha_{11}p_{12}\hat{\delta}_{2}  & \alpha_{11}p_{13}\hat{\delta}_{3}  \\
%     \alpha_{21}p_{21}\hat{\delta}_{1}  & \alpha_{21}p_{22}\hat{\delta}_{2}  & \alpha_{21}p_{23}\hat{\delta}_{3}  \\
%     \alpha_{31}p_{31}\hat{\delta}_{1}  & \alpha_{31}p_{32}\hat{\delta}_{2}  & \alpha_{31}p_{33}\hat{\delta}_{3}  
% \end{bmatrix}
% \end{equation*}
% \begin{equation*}
% \begin{bmatrix}
%     1 & 1 & 1
% \end{bmatrix} 
% \begin{bmatrix}
%     a_{11}  & a_{12}  & a_{13}  \\
%     a_{21}  & a_{22}  & a_{23}  \\
%     a_{31}  & a_{32}  & a_{33}  
% \end{bmatrix} = 
% \begin{bmatrix}
%     a_{11}  + a_{21}  + a_{31}  \\
%     a_{12}  + a_{22}  + a_{31}  \\
%     a_{13}  + a_{23}  + a_{33}  
% \end{bmatrix}
% \end{equation*}
% \begin{align*}
% &\begin{bmatrix}
%     p_{11}(1-\hat{\delta}_{1}) & p_{12}(1-\hat{\delta}_{2}) & p_{13}(1-\hat{\delta}_{3}) \\
%     p_{21}(1-\hat{\delta}_{1}) & p_{22}(1-\hat{\delta}_{2}) & p_{23}(1-\hat{\delta}_{3}) \\
%     p_{31}(1-\hat{\delta}_{1}) & p_{32}(1-\hat{\delta}_{2}) & p_{33}(1-\hat{\delta}_{3}) 
% \end{bmatrix}
% \begin{bmatrix}
%     p_{11}\hat{\delta}_{1}  & p_{12}\hat{\delta}_{2}  & p_{13}\hat{\delta}_{3}  \\
%     p_{21}\hat{\delta}_{1}  & p_{22}\hat{\delta}_{2}  & p_{23}\hat{\delta}_{3}  \\
%     p_{31}\hat{\delta}_{1}  & p_{32}\hat{\delta}_{2}  & p_{33}\hat{\delta}_{3}  
% \end{bmatrix}
% \begin{bmatrix}
%     1 \\ 1 \\ 1
% \end{bmatrix}=\\
% &\begin{bmatrix}
%     p_{11}(1-\hat{\delta}_{1}) & p_{12}(1-\hat{\delta}_{2}) & p_{13}(1-\hat{\delta}_{3}) \\
%     p_{21}(1-\hat{\delta}_{1}) & p_{22}(1-\hat{\delta}_{2}) & p_{23}(1-\hat{\delta}_{3}) \\
%     p_{31}(1-\hat{\delta}_{1}) & p_{32}(1-\hat{\delta}_{2}) & p_{33}(1-\hat{\delta}_{3}) 
% \end{bmatrix}
% \begin{bmatrix}
%     p_{11}\hat{\delta}_{1}  + p_{12}\hat{\delta}_{2}  + p_{13}\hat{\delta}_{3} \\
%     p_{21}\hat{\delta}_{1}  + p_{22}\hat{\delta}_{2}  + p_{23}\hat{\delta}_{3} \\
%     p_{31}\hat{\delta}_{1}  + p_{32}\hat{\delta}_{2}  + p_{33}\hat{\delta}_{3} 
% \end{bmatrix}=\\
% &\begin{bmatrix}
%     p_{11}(1-\hat{\delta}_{1})(p_{11}\hat{\delta}_{1}  + p_{12}\hat{\delta}_{2}  + p_{13}\hat{\delta}_{3}) \textcolor{violet}{+}
%     p_{12}(1-\hat{\delta}_{2})(p_{21}\hat{\delta}_{1}  + p_{22}\hat{\delta}_{2}  + p_{23}\hat{\delta}_{3}) \textcolor{violet}{+}
%     p_{13}(1-\hat{\delta}_{3})(p_{31}\hat{\delta}_{1}  + p_{32}\hat{\delta}_{2}  + p_{33}\hat{\delta}_{3}) \\
%     p_{21}(1-\hat{\delta}_{1})(p_{11}\hat{\delta}_{1}  + p_{12}\hat{\delta}_{2}  + p_{13}\hat{\delta}_{3}) \textcolor{violet}{+}    
%     p_{22}(1-\hat{\delta}_{2})(p_{21}\hat{\delta}_{1}  + p_{22}\hat{\delta}_{2}  + p_{23}\hat{\delta}_{3}) \textcolor{violet}{+}
%     p_{23}(1-\hat{\delta}_{3})(p_{31}\hat{\delta}_{1}  + p_{32}\hat{\delta}_{2}  + p_{33}\hat{\delta}_{3}) \\
%     p_{31}(1-\hat{\delta}_{1})(p_{11}\hat{\delta}_{1}  + p_{12}\hat{\delta}_{2}  + p_{13}\hat{\delta}_{3}) \textcolor{violet}{+}
%     p_{32}(1-\hat{\delta}_{2})(p_{21}\hat{\delta}_{1}  + p_{22}\hat{\delta}_{2}  + p_{23}\hat{\delta}_{3}) \textcolor{violet}{+}
%     p_{33}(1-\hat{\delta}_{3})(p_{31}\hat{\delta}_{1}  + p_{32}\hat{\delta}_{2}  + p_{33}\hat{\delta}_{3}) \\
% \end{bmatrix}
% \end{align*}
% \begin{align*}
% &\begin{bmatrix}
%     p_{11}(1-\hat{\delta}_{1}) & p_{12}(1-\hat{\delta}_{2}) & p_{13}(1-\hat{\delta}_{3}) \\
%     p_{21}(1-\hat{\delta}_{1}) & p_{22}(1-\hat{\delta}_{2}) & p_{23}(1-\hat{\delta}_{3}) \\
%     p_{31}(1-\hat{\delta}_{1}) & p_{32}(1-\hat{\delta}_{2}) & p_{33}(1-\hat{\delta}_{3}) 
% \end{bmatrix}
% \begin{bmatrix}
%     \alpha_{11} & 0 & 0 \\
%     0 & \alpha_{21} & 0 \\
%     0 & 0 & \alpha_{31} \\
% \end{bmatrix}
% \begin{bmatrix}
%     p_{11}\hat{\delta}_{1}  + p_{12}\hat{\delta}_{2}  + p_{13}\hat{\delta}_{3} \\
%     p_{21}\hat{\delta}_{1}  + p_{22}\hat{\delta}_{2}  + p_{23}\hat{\delta}_{3} \\
%     p_{31}\hat{\delta}_{1}  + p_{32}\hat{\delta}_{2}  + p_{33}\hat{\delta}_{3} 
% \end{bmatrix}=\\
% &\begin{bmatrix}
%     p_{11}(1-\hat{\delta}_{1})\alpha_{11}(p_{11}\hat{\delta}_{1}  + p_{12}\hat{\delta}_{2}  + p_{13}\hat{\delta}_{3}) \textcolor{violet}{+}
%     p_{12}(1-\hat{\delta}_{2})\alpha_{21}(p_{21}\hat{\delta}_{1}  + p_{22}\hat{\delta}_{2}  + p_{23}\hat{\delta}_{3}) \textcolor{violet}{+}
%     p_{13}(1-\hat{\delta}_{3})\alpha_{31}(p_{31}\hat{\delta}_{1}  + p_{32}\hat{\delta}_{2}  + p_{33}\hat{\delta}_{3}) \\
%     p_{21}(1-\hat{\delta}_{1})\alpha_{11}(p_{11}\hat{\delta}_{1}  + p_{12}\hat{\delta}_{2}  + p_{13}\hat{\delta}_{3}) \textcolor{violet}{+}    
%     p_{22}(1-\hat{\delta}_{2})\alpha_{21}(p_{21}\hat{\delta}_{1}  + p_{22}\hat{\delta}_{2}  + p_{23}\hat{\delta}_{3}) \textcolor{violet}{+}
%     p_{23}(1-\hat{\delta}_{3})\alpha_{31}(p_{31}\hat{\delta}_{1}  + p_{32}\hat{\delta}_{2}  + p_{33}\hat{\delta}_{3}) \\
%     p_{31}(1-\hat{\delta}_{1})\alpha_{11}(p_{11}\hat{\delta}_{1}  + p_{12}\hat{\delta}_{2}  + p_{13}\hat{\delta}_{3}) \textcolor{violet}{+}
%     p_{32}(1-\hat{\delta}_{2})\alpha_{21}(p_{21}\hat{\delta}_{1}  + p_{22}\hat{\delta}_{2}  + p_{23}\hat{\delta}_{3}) \textcolor{violet}{+}
%     p_{33}(1-\hat{\delta}_{3})\alpha_{31}(p_{31}\hat{\delta}_{1}  + p_{32}\hat{\delta}_{2}  + p_{33}\hat{\delta}_{3}) \\
% \end{bmatrix}
% \end{align*}
% \vspace{3mm}
% \begin{equation*}
% \begin{bmatrix}
%     \alpha_{11}  & \alpha_{12}   \\
%     \alpha_{21}  & \alpha_{22}   \\
%     \alpha_{31}  & \alpha_{32}   
% \end{bmatrix}
% \begin{bmatrix}
%     1 \\ 0
% \end{bmatrix} = 
% \begin{bmatrix}
%     \alpha_{11}  \\
%     \alpha_{21}  \\
%     \alpha_{31}  
% \end{bmatrix}
% \end{equation*}
% \end{figure*}

% \begin{figure*}[ht]
% \raggedright
% Another helpful expression follows.
% \begin{align}\label{eq:proof4}
% \begin{aligned}
% &\mathbb{P}(\eta_{\tau_{(m+1)}}^{} = (n,\hat{s}_{\nu}) \mid \hat{\theta}_{\tau_{(m)}-1} = \hat{s}_{\mu}) = \\
% & \labelrel={proof4:1} \mathbb{P}(\delta_{\tau_{(m)+1+n}}=1, \hat{\theta}_{\tau_{(m)}+n} = \hat{s}_{\nu}, \delta_{\tau_{(m)}+t_{1}}=0 ~ \forall t_{1}\in\mathbb{Z}^{+} : t_{1}\leq n \mid \hat{\theta}_{\tau_{(m)}-1} = \hat{s}_{\mu}) = \\
% &\labelrel={proof4:2} \frac{\mathbb{P}(\delta_{\tau_{(m)+1+n}}=1, \hat{\theta}_{\tau_{(m)}+n} = \hat{s}_{\nu}, \delta_{\tau_{(m)}+t_{1}}=0 ~ \forall t_{1}\in\mathbb{Z}^{+} : t_{1}\leq n, \hat{\theta}_{\tau_{(m)}-1} = \hat{s}_{\mu})}{\mathbb{P}(\hat{\theta}_{\tau_{(m)}-1} = \hat{s}_{\mu})}, \\
% %%%%%%%%%%%%%%%%%%%%%%%%%%%%%%%%%%%%%%%%%%%%%%%%
% \end{aligned}
% \end{align}
% where 
% \begin{itemize}
%     \item \eqref{proof4:1} uses \eqref{eq:proof0}: \eqref{proof0:1}--\eqref{proof0:5};
%     \item \eqref{proof4:2} uses the conditional probability definition.%;
% \end{itemize}
% The nominator of \eqref{eq:proof4} can be transformed as follows.
% \begin{align}\label{eq:proof5}
% \begin{aligned}
% & \mathbb{P}(\delta_{\tau_{(m)+1+n}}=1, \hat{\theta}_{\tau_{(m)}+n} = \hat{s}_{\nu}, \delta_{\tau_{(m)}+t_{1}}=0 ~ \forall t_{1}\in\mathbb{Z}^{+} : t_{1}\leq n, \hat{\theta}_{\tau_{(m)}-1} = \hat{s}_{\mu}) = \\
% &\labelrel={proof5:1}\sum_{i=1}^{N}\sum_{j=1}^{N}\sum_{i_1=1}^{N}\cdots\sum_{i_n=1}^{N}\sum_{i_{n+1}=1}^{N} \,\mathbb{P}(\delta_{\tau_{(m)+1+n}}=1, \theta_{\tau_{(m)}+1+n}=s_{i_{n+1}}, \hat{\theta}_{\tau_{(m)}+n} = \hat{s}_{\nu}, \delta_{\tau_{(m)}+n}=0, \theta_{\tau_{(m)}+n}=s_{i_{n}},\dots, \\
% & \qquad\qquad\quad\! 
% \delta_{\tau_{(m)}+1}=0,\theta_{\tau_{(m)}+1}=s_{i_{1}},\theta_{\tau_{(m)}}=s_j,\hat{\theta}_{\tau_{(m)}-1} = \hat{s}_{\mu},\theta_{\tau_{(m)}-1}=s_i) = \\
% %%%%%%%%%%%%%%%%%%%%%%%%%%%%%%%%%%%%%%%%%%%%%%%%
% &\labelrel={proof5:2} \sum_{i=1}^{N}\sum_{j=1}^{N}\sum_{i_1=1}^{N}\cdots\sum_{i_n=1}^{N}\sum_{i_{n+1}=1}^{N} \,\mathbb{P}(\delta_{\tau_{(m)+1+n}}=1 \mid \theta_{\tau_{(m)}+1+n}=s_{i_{n+1}}, \hat{\theta}_{\tau_{(m)}+n} = \hat{s}_{\nu}, \delta_{\tau_{(m)}+n}=0, \theta_{\tau_{(m)}+n}=s_{i_{n}},\dots, \\
% & \qquad\qquad\quad\! \delta_{\tau_{(m)}+1}=0,\theta_{\tau_{(m)}+1}=s_{i_{1}},\theta_{\tau_{(m)}}=s_j,\hat{\theta}_{\tau_{(m)}-1} = \hat{s}_{\mu},\theta_{\tau_{(m)}-1}=s_i) \cdot \\
% & \qquad\quad~~~\! \mathbb{P}(\theta_{\tau_{(m)}+1+n}=s_{i_{n+1}} \mid \hat{\theta}_{\tau_{(m)}+n} = \hat{s}_{\nu}, \delta_{\tau_{(m)}+n}=0, \theta_{\tau_{(m)}+n}=s_{i_{n}},\dots,\delta_{\tau_{(m)}+1}=0,\theta_{\tau_{(m)}+1}=s_{i_{1}}, \\
% & \qquad\qquad\quad\! \theta_{\tau_{(m)}}=s_j,\hat{\theta}_{\tau_{(m)}-1} = \hat{s}_{\mu},\theta_{\tau_{(m)}-1}=s_i)\cdot \\
% & \qquad\quad~~~\! \mathbb{P}(\hat{\theta}_{\tau_{(m)}+n} = \hat{s}_{\nu} \mid \delta_{\tau_{(m)}+n}=0, \theta_{\tau_{(m)}+n}=s_{i_{n}},\dots,\delta_{\tau_{(m)}+1}=0,\theta_{\tau_{(m)}+1}=s_{i_{1}},\theta_{\tau_{(m)}}=s_j, \\
% & \qquad\qquad\quad\! \hat{\theta}_{\tau_{(m)}-1} = \hat{s}_{\mu},\theta_{\tau_{(m)}-1}=s_i)\cdot \\
% & \qquad\quad~~~\! \mathbb{P}(\delta_{\tau_{(m)}+n}=0 \mid \theta_{\tau_{(m)}+n}=s_{i_{n}},\dots,\delta_{\tau_{(m)}+1}=0,\theta_{\tau_{(m)}+1}=s_{i_{1}},\theta_{\tau_{(m)}}=s_j, \hat{\theta}_{\tau_{(m)}-1} = \hat{s}_{\mu},\theta_{\tau_{(m)}-1}=s_i)\cdot \\
% & \qquad\quad~~~\!
% \mathbb{P}(\theta_{\tau_{(m)}+n}=s_{i_{n}}\mid\dots,\delta_{\tau_{(m)}+1}=0,\theta_{\tau_{(m)}+1}=s_{i_{1}},\theta_{\tau_{(m)}}=s_j, \hat{\theta}_{\tau_{(m)}-1} = \hat{s}_{\mu},\theta_{\tau_{(m)}-1}=s_i)\cdots \\
% & \qquad\quad~~~\!
% \mathbb{P}(\delta_{\tau_{(m)}+1}=0 \mid \theta_{\tau_{(m)}+1}=s_{i_{1}},\theta_{\tau_{(m)}}=s_j, \hat{\theta}_{\tau_{(m)}-1} = \hat{s}_{\mu},\theta_{\tau_{(m)}-1}=s_i)\cdot \\
% & \qquad\quad~~~\!
% \mathbb{P}(\theta_{\tau_{(m)}+1}=s_{i_{1}} \mid \theta_{\tau_{(m)}}=s_j, \hat{\theta}_{\tau_{(m)}-1} = \hat{s}_{\mu},\theta_{\tau_{(m)}-1}=s_i)\cdot \\
% & \qquad\quad~~~\!
% \mathbb{P}(\theta_{\tau_{(m)}}=s_j \mid \hat{\theta}_{\tau_{(m)}-1} = \hat{s}_{\mu},\theta_{\tau_{(m)}-1}=s_i)\cdot 
% \mathbb{P}(\hat{\theta}_{\tau_{(m)}-1} = \hat{s}_{\mu}\mid\theta_{\tau_{(m)}-1}=s_i) \cdot 
% \mathbb{P}(\theta_{\tau_{(m)}-1}=s_i)\\
% %%%%%%%%%%%%%%%%%%%%%%%%%%%%%%%%%%%%%%%%%%%%%%%%
% &\labelrel={proof5:3} \sum_{i=1}^{N}\sum_{j=1}^{N}\sum_{i_1=1}^{N}\cdots\sum_{i_n=1}^{N}\sum_{i_{n+1}=1}^{N} \,\mathbb{P}(\delta_{\tau_{(m)+1+n}}=1 \mid \theta_{\tau_{(m)}+1+n}=s_{i_{n+1}}) \cdot \\
% & \qquad\quad~~~\! \mathbb{P}(\theta_{\tau_{(m)}+1+n}=s_{i_{n+1}} \mid \hat{\theta}_{\tau_{(m)}+n} = \hat{s}_{\nu}, \delta_{\tau_{(m)}+n}=0, \theta_{\tau_{(m)}+n}=s_{i_{n}},\dots,\delta_{\tau_{(m)}+1}=0,\theta_{\tau_{(m)}+1}=s_{i_{1}}, \\
% & \qquad\qquad\quad\! \theta_{\tau_{(m)}}=s_j,\hat{\theta}_{\tau_{(m)}-1} = \hat{s}_{\mu},\theta_{\tau_{(m)}-1}=s_i)\cdot \\
% & \qquad\quad~~~\! \mathbb{P}(\hat{\theta}_{\tau_{(m)}+n} = \hat{s}_{\nu} \mid \delta_{\tau_{(m)}+n}=0, \theta_{\tau_{(m)}+n}=s_{i_{n}},\dots,\delta_{\tau_{(m)}+1}=0,\theta_{\tau_{(m)}+1}=s_{i_{1}},\theta_{\tau_{(m)}}=s_j, \\
% & \qquad\qquad\quad\! \hat{\theta}_{\tau_{(m)}-1} = \hat{s}_{\mu},\theta_{\tau_{(m)}-1}=s_i)\cdot \\
% & \qquad\quad~~~\! \mathbb{P}(\delta_{\tau_{(m)}+n}=0 \mid \theta_{\tau_{(m)}+n}=s_{i_{n}})\cdot \\
% & \qquad\quad~~~\!\mathbb{P}(\theta_{\tau_{(m)}+n}=s_{i_{n}}\mid\dots,\delta_{\tau_{(m)}+1}=0,\theta_{\tau_{(m)}+1}=s_{i_{1}},\theta_{\tau_{(m)}}=s_j, \hat{\theta}_{\tau_{(m)}-1} = \hat{s}_{\mu},\theta_{\tau_{(m)}-1}=s_i)\cdots \\
% & \qquad\quad~~~\!
% \mathbb{P}(\delta_{\tau_{(m)}+1}=0 \mid \theta_{\tau_{(m)}+1}=s_{i_{1}})\cdot \\
% & \qquad\quad~~~\!
% \mathbb{P}(\theta_{\tau_{(m)}+1}=s_{i_{1}} \mid \theta_{\tau_{(m)}}=s_j, \hat{\theta}_{\tau_{(m)}-1} = \hat{s}_{\mu},\theta_{\tau_{(m)}-1}=s_i)\cdot \\
% & \qquad\quad~~~\!
% \mathbb{P}(\theta_{\tau_{(m)}}=s_j \mid \hat{\theta}_{\tau_{(m)}-1} = \hat{s}_{\mu},\theta_{\tau_{(m)}-1}=s_i)\cdot 
% \mathbb{P}(\hat{\theta}_{\tau_{(m)}-1} = \hat{s}_{\mu}\mid\theta_{\tau_{(m)}-1}=s_i) \cdot 
% \mathbb{P}(\theta_{\tau_{(m)}-1}=s_i),
% %%%%%%%%%%%%%%%%%%%%%%%%%%%%%%%%%%%%%%%%%%%%%%%%
% \end{aligned}
% \end{align}
% where 
% \begin{itemize}
%     \item \eqref{proof5:1} is due to the law of total probability and commutative property of set intersection;
%     \item \eqref{proof5:2} follows from the chain rule of probability;
%     \item \eqref{proof5:3} is a consequence of the independence of $\delta_{k}$ from $\delta_{k-t_{1}}$, $\theta_{k-t_{1}}$, and $\hat{\theta}_{k}$ for all $t_{1}\geq 1$.
% \end{itemize}
% \end{figure*}
% \begin{figure*}[ht]
% \raggedright
% Further derivation of the nominator of \eqref{eq:proof4} is as follows.
% \begin{align}\label{eq:proof6}
% \begin{aligned}
% &\eqref{eq:proof5} 
% %%%%%%%%%%%%%%%%%%%%%%%%%%%%%%%%%%%%%%%%%%%%%%%%
% \labelrel={proof6:1} \sum_{i=1}^{N}\sum_{j=1}^{N}\sum_{i_1=1}^{N}\cdots\sum_{i_n=1}^{N}\sum_{i_{n+1}=1}^{N} \,\mathbb{P}(\delta_{\tau_{(m)+1+n}}=1 \mid \theta_{\tau_{(m)}+1+n}=s_{i_{n+1}}) \cdot \\
% & \qquad\quad~~~\! \mathbb{P}(\theta_{\tau_{(m)}+1+n}=s_{i_{n+1}} \mid \theta_{\tau_{(m)}+n}=s_{i_{n}},\dots,\theta_{\tau_{(m)}+1}=s_{i_{1}}, \theta_{\tau_{(m)}}=s_j,\theta_{\tau_{(m)}-1}=s_i)\cdot \\
% & \qquad\quad~~~\! \mathbb{P}(\hat{\theta}_{\tau_{(m)}+n} = \hat{s}_{\nu} \mid \delta_{\tau_{(m)}+n}=0, \theta_{\tau_{(m)}+n}=s_{i_{n}},\dots,\delta_{\tau_{(m)}+1}=0,\theta_{\tau_{(m)}+1}=s_{i_{1}},\theta_{\tau_{(m)}}=s_j, \\
% & \qquad\qquad\quad\! \hat{\theta}_{\tau_{(m)}-1} = \hat{s}_{\mu},\theta_{\tau_{(m)}-1}=s_i)\cdot \\
% & \qquad\quad~~~\! \mathbb{P}(\delta_{\tau_{(m)}+n}=0 \mid \theta_{\tau_{(m)}+n}=s_{i_{n}})\cdot 
% \mathbb{P}(\theta_{\tau_{(m)}+n}=s_{i_{n}}\mid\dots,\theta_{\tau_{(m)}+1}=s_{i_{1}},\theta_{\tau_{(m)}}=s_j, \theta_{\tau_{(m)}-1}=s_i)\cdots \\
% & \qquad\quad~~~\!
% \mathbb{P}(\delta_{\tau_{(m)}+1}=0 \mid \theta_{\tau_{(m)}+1}=s_{i_{1}})\cdot 
% \mathbb{P}(\theta_{\tau_{(m)}+1}=s_{i_{1}} \mid \theta_{\tau_{(m)}}=s_j,\theta_{\tau_{(m)}-1}=s_i)\cdot \\
% & \qquad\quad~~~\!
% \mathbb{P}(\theta_{\tau_{(m)}}=s_j \mid \theta_{\tau_{(m)}-1}=s_i)\cdot 
% \mathbb{P}(\hat{\theta}_{\tau_{(m)}-1} = \hat{s}_{\mu}\mid\theta_{\tau_{(m)}-1}=s_i) \cdot 
% \mathbb{P}(\theta_{\tau_{(m)}-1}=s_i) = \\
% %%%%%%%%%%%%%%%%%%%%%%%%%%%%%%%%%%%%%%%%%%%%%%%%
% & \labelrel={proof6:2} \sum_{i=1}^{N}\sum_{j=1}^{N}\sum_{i_1=1}^{N}\cdots\sum_{i_n=1}^{N}\sum_{i_{n+1}=1}^{N} \,\mathbb{P}(\delta_{\tau_{(m)+1+n}}=1 \mid \theta_{\tau_{(m)}+1+n}=s_{i_{n+1}}) \cdot \\
% & \qquad\quad~~~\! \mathbb{P}(\theta_{\tau_{(m)}+1+n}=s_{i_{n+1}} \mid \theta_{\tau_{(m)}+n}=s_{i_{n}},\dots,\theta_{\tau_{(m)}+1}=s_{i_{1}}, \theta_{\tau_{(m)}}=s_j,\theta_{\tau_{(m)}-1}=s_i)\cdot \\
% & \qquad\quad~~~\! \mathbb{P}(\hat{\theta}_{\tau_{(m)}+n} = \hat{s}_{\nu} \mid \theta_{\tau_{(m)}+n}=s_{i_{n}})\cdot \\
% & \qquad\quad~~~\! \mathbb{P}(\delta_{\tau_{(m)}+n}=0 \mid \theta_{\tau_{(m)}+n}=s_{i_{n}})\cdot 
% \mathbb{P}(\theta_{\tau_{(m)}+n}=s_{i_{n}}\mid\dots,\theta_{\tau_{(m)}+1}=s_{i_{1}},\theta_{\tau_{(m)}}=s_j, \theta_{\tau_{(m)}-1}=s_i)\cdots \\
% & \qquad\quad~~~\!
% \mathbb{P}(\delta_{\tau_{(m)}+1}=0 \mid \theta_{\tau_{(m)}+1}=s_{i_{1}})\cdot 
% \mathbb{P}(\theta_{\tau_{(m)}+1}=s_{i_{1}} \mid \theta_{\tau_{(m)}}=s_j,\theta_{\tau_{(m)}-1}=s_i)\cdot \\
% & \qquad\quad~~~\!
% \mathbb{P}(\theta_{\tau_{(m)}}=s_j \mid \theta_{\tau_{(m)}-1}=s_i)\cdot 
% \mathbb{P}(\hat{\theta}_{\tau_{(m)}-1} = \hat{s}_{\mu}\mid\theta_{\tau_{(m)}-1}=s_i) \cdot 
% \mathbb{P}(\theta_{\tau_{(m)}-1}=s_i) = \\
% %%%%%%%%%%%%%%%%%%%%%%%%%%%%%%%%%%%%%%%%%%%%%%%%
% & \labelrel={proof6:3} \sum_{i=1}^{N}\sum_{j=1}^{N}\sum_{i_1=1}^{N}\cdots\sum_{i_n=1}^{N}\sum_{i_{n+1}=1}^{N} \,\hat{\delta}_{i_{n+1}} \mathbb{P}(\theta_{\tau_{(m)}+1+n}=s_{i_{n+1}} \mid \theta_{\tau_{(m)}+n}=s_{i_{n}}) \, \alpha_{i_{n}\nu} \, (1-\hat{\delta}_{i_n}) \cdot \\
% & \qquad\quad~~~\! \mathbb{P}(\theta_{\tau_{(m)}+n}=s_{i_{n}}\mid \theta_{\tau_{(m)}+n-1}=s_{i_{n-1}})\cdots (1-\hat{\delta}_{i_1})
% \mathbb{P}(\theta_{\tau_{(m)}+1}=s_{i_{1}} \mid \theta_{\tau_{(m)}}=s_j)\,p_{ij}\alpha_{i\mu} \mathbb{P}(\theta_{\tau_{(m)}-1}=s_i) = \\
% %%%%%%%%%%%%%%%%%%%%%%%%%%%%%%%%%%%%%%%%%%%%%%%%
% & \labelrel={proof6:4} \sum_{i=1}^{N}\sum_{j=1}^{N}\sum_{i_1=1}^{N}\cdots\sum_{i_n=1}^{N}\sum_{i_{n+1}=1}^{N} \hat{\delta}_{i_{n+1}} p_{i_{n} i_{n+1}} \, \alpha_{i_{n}\nu} (1-\hat{\delta}_{i_n}) p_{i_{n-1} i_{n}} \cdots (1-\hat{\delta}_{i_1}) p_{j i_{1}} p_{ij} \alpha_{i\mu} \pi_{i}(\tau_{(m)}-1) = \\
% %%%%%%%%%%%%%%%%%%%%%%%%%%%%%%%%%%%%%%%%%%%%%%%%
% %%%
% %%%%%%%%%%%%%%%%%%%%%%%%%%%%%%%%%%%%%%%%%%%%%%%%
% & \labelrel={proof6:5} \sum_{i=1}^{N}\sum_{j=1}^{N}\sum_{i_1=1}^{N}\cdots\sum_{i_n=1}^{N}\sum_{i_{n+1}=1}^{N} 
% \alpha_{i\mu} \pi_{i}(\tau_{(m)}-1) p_{ij} p_{j i_{1}} (1-\hat{\delta}_{i_1}) \cdots
% p_{i_{n-1} i_{n}} (1-\hat{\delta}_{i_n}) \alpha_{i_{n}\nu} p_{i_{n} i_{n+1}} \hat{\delta}_{i_{n+1}}  = \\
% %%%%%%%%%%%%%%%%%%%%%%%%%%%%%%%%%%%%%%%%%%%%%%%%
% & \labelrel={proof6:6} \hat{e}_{\mu}^{\top} P_{e}^{\top} P_{\pi}(\tau{(m)}-1) P_{c} P_{f}^{n} P_{\alpha}(\nu) P_{s} \mathbf{1}, \\
% %%%%%%%%%%%%%%%%%%%%%%%%%%%%%%%%%%%%%%%%%%%%%%%%
% \end{aligned}
% \end{align}
% where 
% \begin{itemize}
%     \item \eqref{proof6:1} relies on the independence of $\theta_{k}$ from $\hat{\theta}_{k-t_2}$ and $\delta_{k-t_2}$ for all $t_2\geq 0$;
%     \item \eqref{proof6:2} applies the independence of $\hat{\theta}_{k}$ from $\theta_{k-t_3}$, $\hat{\theta}_{k-t_3}$, and $\delta_{k-t_4}$ for all $t_3\geq 1$ and $t_4\geq 0$;
%     \item \eqref{proof6:3} uses the Markov property, \eqref{eq:p-delta}, and \eqref{eq:alpha};
%     \item \eqref{proof6:4} is due to \eqref{eq:p-ij} and \eqref{eq:pik};
%     \item \eqref{proof6:5} applies the commutative property of the scalar product;
%     \item \eqref{proof6:6} applies \eqref{eq:Ps}, \eqref{eq:epm}, and the definition of the emission probability matrix.
% \end{itemize}
% \end{figure*}
% \begin{figure*}[ht]
% \raggedright
% The nominator of \eqref{eq:proof4} has the following expression.
% \begin{align}\label{eq:proof7}
% \begin{aligned}
% &\mathbb{P}(\hat{\theta}_{\tau_{(m)}-1} = \hat{s}_{\mu}) = \\
% &\labelrel={proof7:1} \sum_{i=1}^{N} \,\mathbb{P}(\hat{\theta}_{\tau_{(m)}-1} = \hat{s}_{\mu},\theta_{\tau_{(m)}-1}=s_i ) =\\
% &\labelrel={proof7:2} \sum_{i=1}^{N}\,\mathbb{P}(\hat{\theta}_{\tau_{(m)}-1} = \hat{s}_{\mu} \mid \theta_{\tau_{(m)}-1}=s_i )\cdot\mathbb{P}(\theta_{\tau_{(m)}-1}=s_i ) = \\
% &\labelrel={proof7:3} \sum_{i=1}^{N} \alpha_{i\mu}\pi_{i}(\tau_{(m)}-1)=\\
% &\labelrel={proof7:4} \hat{e}_{\mu}^{\top} P_{e}^{\top} P_{\pi}(\tau{(m)}-1)\mathbf{1}.
% \end{aligned}
% \end{align}
% where the equalities hold for the following reasons:
% \begin{itemize}
%     \item \eqref{proof7:1} is due to the law of total probability;
%     \item \eqref{proof7:2} follows from the chain rule of probability;
%     \item \eqref{proof7:3} applies \eqref{eq:p-ij} and \eqref{eq:pik};
%     \item \eqref{proof7:4} \eqref{eq:epm} and the definition of the emission probability matrix.
% \end{itemize}
% \end{figure*}
% \begin{figure*}[ht]
% \raggedright
% From \eqref{eq:proof6} and \eqref{eq:proof7}, we obtain the following expression for \eqref{eq:proof4}.
% \begin{equation}\label{eq:proof8}
%     \mathbb{P}(\eta_{\tau_{(m+1)}}^{} = (n,\hat{s}_{\nu}) \mid \hat{\theta}_{\tau_{(m)}-1} = \hat{s}_{\mu}) =
%     \frac{\hat{e}_{\mu}^{\top} P_{e}^{\top} P_{\pi}(\tau{(m)}-1) P_{c} P_{f}^{n} P_{\alpha}(\nu) P_{s} \mathbf{1}}{\hat{e}_{\mu}^{\top} P_{e}^{\top} P_{\pi}(\tau{(m)}-1)\mathbf{1}} \triangleq \sigma_{(\mu,n,\nu)}^{\tau_{(m)}}.
% \end{equation}
% \end{figure*}
% \begin{figure*}[ht]
% \raggedright
% Notice that under the perfect channel state estimation $M=N$ and $P_e = I_{N}$ so that \eqref{eq:proof8} becomes
% \begin{align}
% \mathbb{P}(\eta&_{\tau_{(m+1)}}^{} = (n,\hat{s}_{\nu}=s_j) \mid \hat{\theta}_{\tau_{(m)}-1} = \hat{s}_{\mu} = s_{i}) =
%     \frac{\pi_{i}(\tau_{(m)}-1)e_{i}^{\top} P_{c} P_{f}^{n} e_{j}e_{j}^{\top} P_{s} \mathbf{1}}{\pi_{i}(\tau_{(m)}-1)e_{i}^{\top}\mathbf{1}} = e_{i}^{\top} P_{c} P_{f}^{n} e_{j}e_{j}^{\top} P_{s} \mathbf{1}.
% \end{align}
% since $\hat{e}_{\mu}^{\top} P_{e}^{\top} = e_{i}^{\top} I_{N} = e_{i}^{\top}$, $e_{i}^{\top} P_{\pi}(\tau{(m)}-1)=e_{i}^{\top}\pi_{i}(\tau_{(m)}-1)=\pi_{i}(\tau_{(m)}-1)e_{i}^{\top}$, $e_{i}^{\top}\mathbf{1}=1$, and $P_{\alpha}(\nu=j)=e_{j}e_{j}^{\top}$.
% \end{figure*}

\begin{figure*}[ht]
\raggedright
\begin{align}\label{eq:cost-tau1}
\begin{aligned}
 & \text{\small $J_{T}^{a}(x_{\tau_{(m)}},\hat{\theta}_{\tau_{(m)}-1}=\hat{s}_{\mu}) = \!
    \min_{\hat{K}_{(\tau_{(m)},\hat{s}_{\mu})}} \Bigg( x_{\tau_{(m)}}^{\top} \Bigg(Q + \mathbb{E} \Big( \sum_{r=1}^{k-\tau_{(m)}} A^{r \top} Q A^{r}  + (A^{k-\tau_{(m)}+1})^{\top} \mathcal{X}_{(k+1,\hat{\theta}_k)} A^{k-\tau_{(m)}+1}\mid \mathcal{I}_{\tau_{(m)}} \Big) \,+ $}\\
    & ~ \text{\small $\hat{K}_{(\tau_{(m)},\hat{s}_{\mu})}^{\top} \bigg( \mathbb{E} \Big( \sum_{r=1}^{k-\tau_{(m)}} \mathit{\Psi}_{(r-1)}^{\top} Q \mathit{\Psi}_{(r-1)} + H^{r \!\top} \! F^{\!\top} \! R F H^{r} + \mathit{\Psi}_{(k-\tau_{(m)})}^{\top} \mathcal{X}_{(k+1,\hat{\theta}_k)} \mathit{\Psi}_{(k-\tau_{(m)})} \mid \mathcal{I}_{\tau_{(m)}} \Big) + F^{\top}\!\! R F \bigg) \hat{K}_{(\tau_{(m)},\hat{s}_{\mu})} \,+ $} \\
    & ~ \text{\small  $\mathbb{E} \Big( (A^{k-\tau_{(m)}+1})^{\top} \mathcal{X}_{(k+1,\hat{\theta}_k)} \mathit{\Psi}_{(k-\tau_{(m)})} \hat{K}_{(\tau_{(m)},\hat{s}_{\mu})} + \hat{K}_{(\tau_{(m)},\hat{s}_{\mu})}^{\top} \mathit{\Psi}_{(k-\tau_{(m)})} ^{\top} \mathcal{X}_{(k+1,\hat{\theta}_k)} A^{k-\tau_{(m)}+1} \,+$} \\
    & ~ \text{\small $ \sum_{r=1}^{k-\tau_{(m)}}\hat{K}_{(\tau_{(m)},\hat{s}_{\mu})}^{\top} \mathit{\Psi}_{(r-1)}^{\top} Q A^{r} +  A^{r \top} Q \mathit{\Psi}_{(r-1)} \hat{K}_{(\tau_{(m)},\hat{s}_{\mu})} 
    \mid \mathcal{I}_{\tau_{(m)}} \Big) \! \Bigg) x_{\tau_{(m)}} \,+$} \\
    & ~ \text{\small $\mathbb{E} \Big( \sum_{\nu=0}^{k-\tau_{(m)}} \mathop{\mathrm{tr}} (A^{\nu \top}  \mathcal{X}_{(k+1,\hat{\theta}_k)} A^{\nu} \Sigma_W) +\sum_{r=1}^{k-\tau_{(m)}} \!\sum_{\nu=0}^{r-1} \mathop{\mathrm{tr}} (A^{\nu \top} \! Q A^{\nu} \Sigma_W) + g_{(k+1,\hat{\theta}_k)} \mid \mathcal{I}_{\tau_{(m)}} \! \Big)\!\Bigg)$}
\end{aligned}
\end{align}
\begin{equation}\label{eq:xik}
    \xi_k \triangleq \max \{0, k+1+L-T \}
\end{equation}
so that $\xi_{T-1}=L$, and $\xi_{k} = 0$ for all $k<T-L$.
\begin{equation}\label{eq:fh-x-t}
    \mathcal{X}_{(T,\hat{s}_{\mu})} = Q,    
\end{equation}
\begin{equation}\label{eq:fh-g-t}
    g_{(T,\hat{s}_{\mu})} = 0. 
\end{equation}
\begin{align}\label{eq:fh-a}
    \mathcal{A}_{(\tau_{(m)},\hat{s}_{\mu})} = Q + 
    \sum_{h=0}^{L-\xi_k} \sum_{\nu=1}^{M} \zeta_{(\mu,h,\nu)}^{\tau_{(m)}} \left(
    (A^{h+1})^{\top} \mathcal{X}_{(\tau_{(m)}+1+h,\hat{s}_{\nu})} A^{h+1} + 
    \sum_{r=1}^{h} A^{r \top} Q A^{r}\right),
\end{align}
\begin{align}\label{eq:fh-b}
    \mathcal{B}_{(\tau_{(m)},\hat{s}_{\mu})} = F^{\!\top} \! R F + 
    \sum_{h=0}^{L-\xi_k} \sum_{\nu=1}^{M} \zeta_{(\mu,h,\nu)}^{\tau_{(m)}} \left(
    \mathit{\Psi}_{(h)}^{\top}  \mathcal{X}_{(\tau_{(m)}+1+h,\hat{s}_{\nu})}  \mathit{\Psi}_{(h)}^{} + 
    \sum_{r=1}^{h} \left(\mathit{\Psi}_{(r-1)}^{\top} Q \mathit{\Psi}_{(r-1)}^{} + H^{r \!\top} \! F^{\!\top} \! R F H^{r} \right)
    \right),
\end{align}
\begin{align}\label{eq:fh-c}
    \mathcal{C}_{(\tau_{(m)},\hat{s}_{\mu})} = 
    \sum_{h=0}^{L-\xi_k} \sum_{\nu=1}^{M} \zeta_{(\mu,h,\nu)}^{\tau_{(m)}} \left(
    \mathit{\Psi}_{(h)}^{\top}  \mathcal{X}_{(\tau_{(m)}+1+h,\hat{s}_{\nu})} A^{h+1} + 
    \sum_{r=1}^{h} \mathit{\Psi}_{(r-1)}^{\top} Q A^{r}
    \right),
\end{align}
\begin{align}\label{eq:fh-gk}
    g_{(\tau_{(m)},\hat{s}_{\mu})} = \sum_{h=0}^{L-\xi_k} \sum_{\nu=1}^{M} \zeta_{(\mu,h,\nu)}^{\tau_{(m)}} \left( g_{(\tau_{(m)}+1+h,\hat{s}_{\nu})} +  
    \sum_{t=0}^{h} \mathop{\mathrm{tr}}(A^{t \top} \mathcal{X}_{(\tau_{(m)}+1+h,\hat{s}_{\nu})} A^{t} \Sigma_W ) + 
    \sum_{r=1}^{h} \sum_{\iota=0}^{r-1} 
    \mathop{\mathrm{tr}}( A^{\iota \top} Q A^{\iota} \Sigma_W )    
    \right).
\end{align}
\end{figure*}
\begin{figure*}[ht]
\raggedright
\begin{align}\label{eq:cost-tau2}
\begin{aligned}
 & J_{T}^{a}(x_{\tau_{(m)}},\hat{\theta}_{\tau_{(m)}-1}=\hat{s}_{\mu}) = 
    x_{\tau_{(m)}}^{\top}\mathcal{A}_{(\tau_{(m)},\hat{s}_{\mu})}x_{\tau_{(m)}}^{} \!+ g_{(\tau_{(m)},\hat{s}_{\mu})} \,+ \\
    & \min_{K_{(\tau_{(m)},\hat{s}_{\mu})}} \left(\! x_{\tau_{(m)}}^{\top} \left( 
    \mathcal{C}_{(\tau_{(m)},\hat{s}_{\mu})}^{\top} K_{(\tau_{(m)},\hat{s}_{\mu})} + K_{(\tau_{(m)},\hat{s}_{\mu})}^{\top} \mathcal{B}_{(\tau_{(m)},\hat{s}_{\mu})} K_{(\tau_{(m)},\hat{s}_{\mu})}^{} + 
    K_{(\tau_{(m)},\hat{s}_{\mu})}^{\top}\mathcal{C}_{(\tau_{(m)},\hat{s}_{\mu})}^{} \right) x_{\tau_{(m)}} \right)
\end{aligned}
\end{align}
\end{figure*}

\begin{figure*}[ht]
\raggedright
\begin{subequations}\label{eq:varphi}
    \begin{equation}\label{eq:varphi-current}
        \hat{\varphi}_{\tau_{(m+1)}} \triangleq (\hat{\theta}_{\tau_{(m+1)}},\Delta_{\tau_{(m+1)}},\hat{\theta}_{\tau_{(m+1)}-\Delta_{\tau_{(m+1)}}-2}) \labelrel={eq:varphi-current-alternative}
        (\hat{\theta}_{\tau_{(m+1)}},\Delta_{\tau_{(m+1)}},\hat{\theta}_{\tau_{(m)}-1}),
    \end{equation}
    \begin{equation}\label{eq:varphi-next}
         \hat{\varphi}_{\tau_{(m+2)}} \triangleq (\hat{\theta}_{\tau_{(m+2)}},\Delta_{\tau_{(m+2)}},\hat{\theta}_{\tau_{(m+2)}-\Delta_{\tau_{(m+2)}}-2}) =
        (\hat{\theta}_{\tau_{(m+2)}},\Delta_{\tau_{(m+2)}},\hat{\theta}_{\tau_{(m+1)}-1}),
    \end{equation}
\end{subequations}
where equality \eqref{eq:varphi-current-alternative} is due to \eqref{eq:tau}.
\begin{align}\label{eq:proofx}
\begin{aligned}
&\mathbb{P}(\hat{\varphi}_{\tau_{(m+2)}}^{} = 
(\hat{s}_{\nu_{1}},n,\hat{s}_{\nu_{0}}) \mid \hat{\varphi}_{\tau_{(m+1)}}^{} = (\hat{s}_{\mu_{1}},\ell,\hat{s}_{\mu_{0}}) = \\
&\labelrel={proofx:1}
\mathbb{P}(\hat{\varphi}_{\tau_{(m+2)}}^{} = 
(\hat{s}_{\nu_{1}},n,\hat{s}_{\nu_{0}}), \tau_{(m+2)} \in \mathcal{T} \mid \hat{\varphi}_{\tau_{(m+1)}}^{} = (\hat{s}_{\mu_{1}},\ell,\hat{s}_{\mu_{0}}), \tau_{(m+1)} \in \mathcal{T}) = \\
&\labelrel={proofx:2}
\mathbb{P}(\mathit{\Delta}_{\tau_{(m+2)}}=n,\hat{\theta}_{\tau_{(m+1)}+\mathit{\Delta}_{\tau_{(m+2)}}+1} = \hat{s}_{\nu_1}, \hat{\theta}_{\tau_{(m+1)}-1} = \hat{s}_{\nu_{0}}, \tau_{(m+2)} \in \mathcal{T} \mid \\
& \qquad~ \mathit{\Delta}_{\tau_{(m+1)}}=\ell,\hat{\theta}_{\tau_{(m)}+\mathit{\Delta}_{\tau_{(m+1)}}+1} = \hat{s}_{\mu_1},\hat{\theta}_{\tau_{(m)}-1} = \hat{s}_{\mu_0},\tau_{(m+1)} \in \mathcal{T}) = \\
&\labelrel={proofx:3}
\mathbb{P}(\mathit{\Delta}_{\tau_{(m+2)}}=n,\hat{\theta}_{\tau_{(m+1)}+n+1} = \hat{s}_{\nu_1}, \hat{\theta}_{\tau_{(m+1)}-1} = \hat{s}_{\nu_{0}}, \delta_{\tau_{(m+1)}+n+1}=1 \mid \\
& \qquad~ \mathit{\Delta}_{\tau_{(m+1)}}=\ell,\hat{\theta}_{\tau_{(m)}+\ell+1} = \hat{s}_{\mu_1},\hat{\theta}_{\tau_{(m)}-1} = \hat{s}_{\mu_0},\delta_{\tau_{(m)}+\ell+1}=1) = \\
&\labelrel={proofx:4}
\mathbb{P}(\mathit{\Delta}_{\tau_{(m+2)}}=n,\hat{\theta}_{\tau_{(m)}+\ell+n+2} = \hat{s}_{\nu_1}, \hat{\theta}_{\tau_{(m)}+\ell} = \hat{s}_{\nu_{0}}, \delta_{\tau_{(m)}+\ell+n+2}=1,\delta_{\tau_{(m)}+\ell+1}=1 \mid \\
& \qquad~ \mathit{\Delta}_{\tau_{(m+1)}}=\ell,\hat{\theta}_{\tau_{(m)}+\ell+1} = \hat{s}_{\mu_1},\hat{\theta}_{\tau_{(m)}-1} = \hat{s}_{\mu_0},\delta_{\tau_{(m)}+\ell+1}=1,\delta_{\tau_{(m)}}=1) = \\
&\labelrel={proofx:5}
\mathbb{P}(\mathit{\Delta}_{\tau_{(m+2)}}=n,\hat{\theta}_{\tau_{(m)}+\ell+n+2} = \hat{s}_{\nu_1}, \hat{\theta}_{\tau_{(m)}+\ell} = \hat{s}_{\nu_{0}}, \delta_{\tau_{(m)}+\ell+n+2}=1 \mid \\
& \qquad~ \mathit{\Delta}_{\tau_{(m+1)}}=\ell,\hat{\theta}_{\tau_{(m)}+\ell+1} = \hat{s}_{\mu_1},\hat{\theta}_{\tau_{(m)}-1} = \hat{s}_{\mu_0},\delta_{\tau_{(m)}+\ell+1}=1,\delta_{\tau_{(m)}}=1) = \\
&\labelrel={proofx:6}
\mathbb{P}(\hat{\theta}_{\tau_{(m)}+\ell} = \hat{s}_{\nu_{0}}, \delta_{\tau_{(m)}+\ell+1+t_{1}}=0 ~ \forall t_{1}\in\mathbb{Z}^{+} : t_{1}\leq n, \delta_{\tau_{(m)}+\ell+n+2}=1, \hat{\theta}_{\tau_{(m)}+\ell+n+2} = \hat{s}_{\nu_1}  \mid\\
& \qquad~ \mathit{\Delta}_{\tau_{(m+1)}}=\ell,\hat{\theta}_{\tau_{(m)}+\ell+1} = \hat{s}_{\mu_1},\hat{\theta}_{\tau_{(m)}-1} = \hat{s}_{\mu_0},\delta_{\tau_{(m)}+\ell+1}=1,\delta_{\tau_{(m)}}=1) = \\
%%%
&\labelrel={proofx:7}
\mathbb{P}(\hat{\theta}_{\tau_{(m)}+\ell} = \hat{s}_{\nu_{0}}, \delta_{\tau_{(m)}+\ell+1+t_{1}}=0 ~ \forall t_{1}\in\mathbb{Z}^{+} : t_{1}\leq n, \delta_{\tau_{(m)}+\ell+n+2}=1,\hat{\theta}_{\tau_{(m)}+\ell+n+2} = \hat{s}_{\nu_1}  \mid \\
& \qquad~ \hat{\theta}_{\tau_{(m)}-1} = \hat{s}_{\mu_0}, \delta_{\tau_{(m)}}=1, \delta_{\tau_{(m)}+t_{2}}=0 ~ \forall t_{2}\in\mathbb{Z}^{+} : t_{2}\leq \ell, \delta_{\tau_{(m)}+\ell+1}=1, \hat{\theta}_{\tau_{(m)}+\ell+1} = \hat{s}_{\mu_1}) = \\
%%%
&\labelrel={proofx:8}
\mathbb{P}(\delta_{\tau_{(m)}+\ell+n+2}=1,\hat{\theta}_{\tau_{(m)}+\ell+n+2} = \hat{s}_{\nu_1}, \delta_{\tau_{(m)}+\ell+1+t_{1}}=0 ~ \forall t_{1}\in\mathbb{Z}^{+} : t_{1}\leq n, \hat{\theta}_{\tau_{(m)}+\ell} = \hat{s}_{\nu_{0}} \mid \\
& \qquad~ \delta_{\tau_{(m)}+\ell+1}=1, \hat{\theta}_{\tau_{(m)}+\ell+1} = \hat{s}_{\mu_1},
\delta_{\tau_{(m)}}=1, \delta_{\tau_{(m)}+t_{2}}=0 ~ \forall t_{2}\in\mathbb{Z}^{+} : t_{2}\leq \ell, \hat{\theta}_{\tau_{(m)}-1} = \hat{s}_{\mu_0}) = \\
&\labelrel={proofx:9}
\mathbb{P}(\delta_{\tau_{(m)}+\ell+n+2}=1,\hat{\theta}_{\tau_{(m)}+\ell+n+2} = \hat{s}_{\nu_1}, \delta_{\tau_{(m)}+\ell+1+t_{1}}=0 ~ \forall t_{1}\in\mathbb{Z}^{+} : t_{1}\leq n \mid \hat{\theta}_{\tau_{(m)}+\ell} = \hat{s}_{\nu_{0}},  \\
& \qquad~ \delta_{\tau_{(m)}+\ell+1}=1, \hat{\theta}_{\tau_{(m)}+\ell+1} = \hat{s}_{\mu_1},
\delta_{\tau_{(m)}}=1, \delta_{\tau_{(m)}+t_{2}}=0 ~ \forall t_{2}\in\mathbb{Z}^{+} : t_{2}\leq \ell, \hat{\theta}_{\tau_{(m)}-1} = \hat{s}_{\mu_0}) \cdot \\
& ~\,\cdot \mathbb{P}(\hat{\theta}_{\tau_{(m)}+\ell} = \hat{s}_{\nu_{0}} \mid \delta_{\tau_{(m)}+\ell+1}=1, \hat{\theta}_{\tau_{(m)}+\ell+1} = \hat{s}_{\mu_1},
\delta_{\tau_{(m)}}=1, \delta_{\tau_{(m)}+t_{2}}=0 ~ \forall t_{2}\in\mathbb{Z}^{+} : t_{2}\leq \ell, \hat{\theta}_{\tau_{(m)}-1} = \hat{s}_{\mu_0}) =\\
% &\labelrel={proofx:10}
% \frac{\mathbb{P}(\delta_{\tau_{(m)+1+n}}=1, \hat{\theta}_{\tau_{(m)}+n} = \hat{s}_{\nu}, \delta_{\tau_{(m)}+t_{1}}=0 ~ \forall t_{1}\in\mathbb{Z}^{+} : t_{1}\leq n, \hat{\theta}_{\tau_{(m)}-1} = \hat{s}_{\mu}, \delta_{\tau_{(m)}}=1)}{\mathbb{P}(\hat{\theta}_{\tau_{(m)}-1} = \hat{s}_{\mu}, \delta_{\tau_{(m)}}=1)}.
\end{aligned}
\end{align}
where commas indicate the intersections of events, and a colon specifies a constraint (such that);
\begin{itemize}
\item the equality~\eqref{proofx:1} is due to \eqref{eq:calT}, i.e., $\forall m \in \mathbb{Z}^{0+}$, it holds that $\tau_{(m)}\in\mathcal{T}$, with $\mathcal{T}\triangleq \left\{ k : \delta_k = 1 \right\}_{k\in \mathbb{Z}^{0+}}$;
\item \eqref{proofx:2} relies on \eqref{eq:varphi};
\item \eqref{proofx:3} substitutes $\mathit{\Delta}_{\tau_{(m+2)}}$ with $n$ in $\hat{\theta}_{\tau_{(m+1)}+\mathit{\Delta}_{\tau_{(m+2)}}+1}$, $\mathit{\Delta}_{\tau_{(m+1)}}$ with $\ell$ in 
$\hat{\theta}_{\tau_{(m)}+\mathit{\Delta}_{\tau_{(m+1)}}+1}$, and applies \eqref{eq:deltatau};
\item \eqref{proofx:4} relies on \eqref{eq:calT}, \eqref{eq:deltatau}, and \eqref{eq:tau+} for $\tau_{(m+1)}$ and $\tau_{(m)}$;
\item \eqref{proofx:5} uses \eqref{eq:chain-rule-special} for $\delta_{\tau_{(m)}+\ell+1}=1$;
\item \eqref{proofx:6} applies \eqref{eq:zl} for $k=\tau_{(m+2)}$;
\item \eqref{proofx:7} applies \eqref{eq:zl} for $k=\tau_{(m+1)}$;
\item \eqref{proofx:8} is due to the commutative property of set intersection;
\item \eqref{proofx:9} relies on the chain rule of the probability.
% \item \eqref{proofx:9} is an immediate consequence of the independence of both $\delta_{k}$ and $\hat{\theta}_{k}$ from $\delta_{k-t_{3}}$, for all $t_{3}\geq 1$;
% \item \eqref{proofx:10} uses the conditional probability definition and the commutative property of the set intersection.
\end{itemize}
\end{figure*}

% \section*{ACKNOWLEDGMENT}
% Put sponsor acknowledgments in the unnumbered footnote on the first page.

%%%%%%%%%%%%%%%%%%%%%%%%%%%%%%%%%%%%%%%%%%%%%%%%%%%%%%%%%%%%%%%%%%%%%%%%%%%%%%%%

% \bibliographystyle{IEEEtran}
% \bibliography{biblio}

\end{document}

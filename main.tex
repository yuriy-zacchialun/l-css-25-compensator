%%%%%%%%%%%%%%%%%%%%%%%%%%%%%%%%%%%%%%%%%%%%%%%%%%%%%%%%%%%%%%%%%%%%%%%%%%%%%%%%
%2345678901234567890123456789012345678901234567890123456789012345678901234567890
%        1         2         3         4         5         6         7         8

\documentclass[journal,twoside,web]{ieeecolor}
\usepackage{orcidlink}
\usepackage{lcsys}
\usepackage{amsmath,amssymb,amsfonts} % assumes amsmath package installed

\newtheorem{assumption}{Assumption}
\newtheorem{theorem}{Theorem}
\newtheorem{corollary}{Corollary}[theorem]
\newtheorem{remark}{Remark}
\newtheorem{lemma}{Lemma}
\newtheorem{proposition}{Proposition}
\newtheorem{definition}{Definition}
\newtheorem{problem}{Problem}

%%%%%%%%%%%%%%%%%%%%%%%%%%%%%%%%%%%%%%%%%%%%%%%%%%%%%%%%%%%%%%%%%%%%%%%%%%%%%%%%
\usepackage{mathtools}
%%%%%%%%%%%%%%%%%%%%%%%%%%%%%%%%%%%%%%%%%%%%%%%%%%%%%%%%%%%%%%%%%%%%%%%%%%%%%%%%
%%%%% For the references above equality and others relationships in equations
%%%%%%%%%%%%%%%%%%%%%%%%%%%%%%%%%%%%%%%%%%%%%%%%%%%%%%%%%%%%%%%%%%%%%%%%%%%%%%%%
\newcounter{relctr} %% <- counter for relations
\everydisplay\expandafter{\the\everydisplay\setcounter{relctr}{0}} %% <- reset every eq
\renewcommand*\therelctr{\alph{relctr}} %% <- label format
%%%%%%%%%%%%%%%%%%%%%%%%%%%%%%%%%%%%%%%%%%%%%%%%%%%%%%%%%%%%%%%%%%%%%%%%%%%%%%%%
\newcommand\labelrel[2]{%
  \begingroup
    \refstepcounter{relctr}%
    \stackrel{\textnormal{(\alph{relctr})}}{\mathstrut{#1}}%
    \originallabel{#2}%
  \endgroup
}
\AtBeginDocument{\let\originallabel\label} %% <- store original definition
%%%%%%%%%%%%%%%%%%%%%%%%%%%%%%%%%%%%%%%%%%%%%%%%%%%%%%%%%%%%%%%%%%%%%%%%%%%%%%%%

\title{Wireless Control With Channel State Detection And Message Dropout Compensation}
% \title{Linear Quadratic Regulation With Wireless Control Message Dropout Compensation}
% \title{Robust Linear Quadratic Regulation Over Polytopic Time-Inhomogeneous Markovian Channels Under Generalized Packet Dropout Compensation}

% \title{Robust LQR Over Polytopic Time-Inhomogeneous Markov Channels Under Generalized Packet Dropout Compensation}

\author{Yuriy \textbf{Zacchia Lun}\,\textsuperscript{\orcidlink{0000-0002-9408-8773}},\,\IEEEmembership{Member, IEEE}, Fortunato \textbf{Santucci}\,\textsuperscript{\orcidlink{0000-0002-0229-6277}},\,\IEEEmembership{Senior Member, IEEE},\\ and Alessandro \textbf{D'Innocenzo}\,\textsuperscript{\orcidlink{0000-0002-5239-0894}},\,\IEEEmembership{Member, IEEE}
\thanks{The authors are with the 
Department of Information Engineering, Computer Science and Mathematics (DISIM), University of L'Aquila, L'Aquila 67100, Italy. (e-mail addresses:
        {yuriy.zacchialun@univaq.it},\\ {fortunato.santucci@univaq.it}, and {alessandro.dinnocenzo@univaq.it})}%
}

\def\BibTeX{{\rm B\kern-.05em{\sc i\kern-.025em b}\kern-.08em
    T\kern-.1667em\lower.7ex\hbox{E}\kern-.125emX}}
\markboth{\journalname, VOL. XX, NO. XX, XXXX}
{Zacchia Lun \MakeLowercase{\textit{et al.}}: Wireless Control With Channel State Detection And Message Dropout Compensation}

\begin{document}

\maketitle
\thispagestyle{empty}
\pagestyle{empty}

%%%%%%%%%%%%%%%%%%%%%%%%%%%%%%%%%%%%%%%%%%%%%%%%%%%%%%%%%%%%%%%%%%%%%%%%%%%%%%%%
\begin{abstract}
This letter addresses.
\end{abstract}
\begin{keywords}
Control over communications, Markov processes.
\end{keywords}
%%%%%%%%%%%%%%%%%%%%%%%%%%%%%%%%%%%%%%%%%%%%%%%%%%%%%%%%%%%%%%%%%%%%%%%%%%%%%%%%
\section{Introduction}\label{sec:intro}
\IEEEPARstart{W}{ireless} networked control systems 

The \textit{contribution} of this letter is twofold:
\begin{itemize}
\item It introduces.
\item It solves.
\end{itemize}
This letter is \emph{organized} as follows. . 

\textit{Notation:} $\mathbb{R}$, $\mathbb{Z}^{+}$, and $\mathbb{Z}^{0+}$ indicate the sets of reals and positive and nonnegative integers. $\mathbb{R}^{n}$ indicates the $n$-dimensional Euclidean space. The symmetric positive definite and positive semi-definite matrices $M$ are denoted by $M\succ 0$ and $M\succeq 0$. Then, $M^{\top}$ indicates the transpose of a matrix $M$. Finally, $\mathbb{P}$ represents the probability of an event, and $\mathbb{E}$ denotes the expected value of a random variable.
\section{System Model and Problem Definition}\label{sec:model}
Consider a discrete-time linear system
\begin{equation}\label{eq:state}
        x_{k+1} = A x_{k} + B u_{k}^{} + w_{k},
\end{equation}
where $\forall k \!\in\! \mathbb{Z}^{0+}$, $x_k\!\in\!\mathbb{R}^{n_x}$ and $u_k^{}\!\in\!\mathbb{R}^{n_u}$ are the system state and control input to actuators, $A$ and $B$ are state and input matrices of appropriate size, and $w_k\!\in\!\mathbb{R}^{n_x}$ is a white Gaussian process noise having zero mean and covariance matrix $\Sigma_w$.

$\{\delta_k\}$ is a binary stochastic process that models the packet loss between the controller and the actuators, and $u_k^c\!\in\!\mathbb{R}^{n_u}$ is the updated desired control input sent by the remote controller. Upon correct delivery, $u_k^{}\!=\!u_k^c$; otherwise, if the desired control input is corrupted or lost, the actuators apply a scaled version of an outdated control input, $u_k^{}\!=\!\phi u_{k-1}^{}$, with $\phi \in [0,1]$.
This generalized approach to control-packet dropout compensation covers both the zero-input ($\phi\!=\!0$) and hold-input ($\phi\!=\!1$) PDCs as notable cases. 
\subsection{Polytopic Markovian channel model}\label{subsec:ptifsmc}
We model the dynamics of the control-packet loss process $\{\delta_k\}$ in the wireless propagation environment as an FSMC. 

This channel has a fixed finite set $\mathcal{S}$ of states $\{s_i\}_{i=1}^{N}$, each characterized by predefined signal-to-interference-plus-noise ratio (SINR) thresholds, which do not vary in time, so the state number remains constant, and each channel state has a specific packet dropout probability. 
A discrete-time Markov chain $\{\theta_k\}$ determines the active channel state, whose packet dropout probability and transition probabilities (TPs) depend on the SINR dynamics. So, we formally define the FSMC as follows.
For all $k\in \mathbb{Z}^{0+}$,
\begin{equation}\label{eq:p-ij}
    \mathbb{P}(\theta_{k} = s_j \mid \theta_{k-1} = s_i) = p_{ij} \geq 0,~ \sum_{j=1}^N p_{ij} = 1,
\end{equation}
\begin{align}\label{eq:p-delta}
\begin{aligned}
    \mathbb{P}(\delta_k = 1 \mid \theta_{k} = s_j) &= \hat{\delta}_{j}(k),\\
    \mathbb{P}(\delta_k = 0 \mid \theta_{k} = s_j) &= 1 - \hat{\delta}_{j}(k).    
\end{aligned}
\end{align}
For notational convenience, group %the
channel state TPs in the transition probability matrix (TPM) $P_{c}^{} \triangleq \left[p_{ij}\right]_{i,j=1}^{N}$
% \begin{equation}\label{eq:Pc}
%     P_{c}^{}(k) \triangleq \left[p_{ij}(k)\right]_{i,j=1}^{N}
% \end{equation}
and the  success (or, conversely, failure) probabilities of delivering a packet at time $k$ in the following matrices:
\begin{equation}\label{eq:Ps}
    P_{s}^{} \triangleq \left[p_{ij}\hat{\delta}_{j}\right]_{i,j=1}^{N},~P_{f} = P_{c} - P_{s}.
\end{equation}
We also indicate the channel state probability at time $k$ as
\begin{equation}\label{eq:pik}
     \mathbb{P}(\theta_{k} = s_i) = \pi_{i}(k).
\end{equation}
Inspired by Costa et al., we use the detector with an emission probability matrix (EPM) $P_{e}^{} \triangleq \left[\alpha_{i\mu}\right]_{i,\mu=1}^{N,M}$, where
\begin{equation}\label{eq:alpha}
   \mathbb{P}(\hat{\theta}_{k} = \hat{s}_{\mu} \mid \theta_{k} = s_i) = \alpha_{i\mu} \geq 0,~ \sum_{i=1}^N \alpha_{i\mu}= 1.
\end{equation}
We then define the following diagonal matrices.
\begin{equation}\label{eq:epm}
     P_{\pi}^{}(k) = \bigoplus_{i=1}^{N} \pi_{i}(k),~P_{\alpha}^{}(\mu) = \bigoplus_{i=1}^{N} \alpha_{i\mu}
\end{equation}
\subsection{System states in packet delivery time instances}
Count the consecutive control message dropouts observed by a controller at time step $k$ in a stochastic variable $\mathit{\Delta}_{k}$.
\begin{equation}\label{eq:z-k}
    \mathit{\Delta}_{k}=(1-\delta_{k-1})(\mathit{\Delta}_{k-1}+1).
\end{equation}
\begin{equation}\label{eq:zl}
    \mathit{\Delta}_{k}=\ell\Leftrightarrow \delta_{k-1-\ell}=1 \land 
	\delta_{k-t}=0 ~ \forall t\in\mathbb{Z}^{+} : t\leq \ell.
\end{equation}
Notably, if $\ell\!=\!0$, then $\left\{t\right\}_{t=1}^{0} \!=\! \emptyset$, meaning that \eqref{eq:zl} becomes $\mathit{\Delta}_{k}\!=\!0\Leftrightarrow \delta_{k-1}\!=\!1$.
Denote by $\mathcal{T}$ a set of time instances in which actuators successfully receive the controller's messages: %. Formally, 
\begin{subequations}\label{eq:tau} 
\begin{equation}\label{eq:calT}
    \mathcal{T} \triangleq \left\{ k : \delta_k = 1 \right\}_{k\in\mathbb{Z}^{0+}} = \{ \tau_{(m)}\}_{m\in\mathbb{Z}^{0+}}.
\end{equation}
From \eqref{eq:z-k}, \eqref{eq:zl}, and \eqref{eq:calT}, for all $m\in\mathbb{Z}^{0+}$,
\begin{equation}\label{eq:deltatau}
    \tau_{(m)} \in \mathcal{T}  \Rightarrow \delta_{\tau_{(m)}} = 1 \Rightarrow \mathit{\Delta}_{\tau_{(m)}+1} = 0,
\end{equation}
\begin{equation}\label{eq:tau+}
    \tau_{(m+1)} = \tau_{(m)} + 1 + \mathit{\Delta}_{\tau_{(m+1)}}.
\end{equation}
\end{subequations}

For the notational convenience, for any $k,n\in\mathbb{Z}^{0+}$, let
\begin{equation}\label{eq:calPsi}
W_{\!(k,n)} \!\triangleq\! \sum\nolimits_{j=0}^{n}A^{n-j} w_{k+j}^{}.
\end{equation}

\subsection{System model for LQR}
To provide the stochastic characterization of the above system dynamics, we group the current duration of a packet error burst and the last known wireless channel state in one augmented discrete state:
$\eta_{k}^{} \!\triangleq\! (\mathit{\Delta}_{k},\theta_{k-1})$. Then, from \eqref{eq:tau},
\begin{subequations}\label{eq:etatau}
\begin{equation}
    \eta_{\tau_{(m)}}^{} = (\mathit{\Delta}_{\tau_{(m)}},\theta_{\tau_{(m)}-1}),
\end{equation}
\begin{equation}
    \eta_{\tau_{(m+1)}}^{} = (\mathit{\Delta}_{\tau_{(m+1)}},\theta_{\tau_{(m)}+\mathit{\Delta}_{\tau_{(m+1)}}}),
\end{equation}
\end{subequations}
where $\mathit{\Delta}_{\tau_{(m+1)}}$ indicates the time interval the transmitted control input may remain active. From the controller's perspective, $\eta_{\tau_{(m)}}^{}$ is known, while $\eta_{\tau_{(m+1)}}^{}$ is a random variable.
\begin{lemma}\label{lemma:z}
For any pair of discrete states \eqref{eq:etatau}, it holds.\\
$e_i$ and $e_j$ denote the $i$th and $j$th column vectors of the standard basis of $\mathbb{R}^{N}$ (all their components are zero except the $i$th for $e_i$ and $j$th for $e_j$, and the nonzero element equals one), and $\mathbf{1}$ indicates the column vector of length $N$ with all components equal one.
\end{lemma}
\begin{remark}\label{rem:1}
TPs are independent of $\ell$, the value of $\mathit{\Delta}_{\tau_{(m)}}$.
\end{remark}

% \section*{ACKNOWLEDGMENT}
% Put sponsor acknowledgments in the unnumbered footnote on the first page.

%%%%%%%%%%%%%%%%%%%%%%%%%%%%%%%%%%%%%%%%%%%%%%%%%%%%%%%%%%%%%%%%%%%%%%%%%%%%%%%%

% \bibliographystyle{IEEEtran}
% \bibliography{biblio}

\end{document}
